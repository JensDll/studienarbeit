\chapter{Reflektion und Ausblick}
Die vorliegende Studienarbeit befasst sich mit der Frage \enquote{Wie kann
  ein Algorithmus entwickelt werden, welcher eine beliebige Nachricht
  in einem Bild versteckt und während\-dessen die Bildqualität nicht sichtbar beeinflusst?}.
Zusätzlich wird den Fragen nachgegangen \enquote{Wie kann die Vertraulichkeit dieser Nachricht
  gesichert werden?} und \enquote{Wie ist der Algorithmus als Teil einer
  größeren Anwendung einzuordnen?}. Um diese Fragen zu beantworten, wurden die
Prinzipien der symmetrischen und asymmetrischen Kryptografie aufgearbeitet
und die Möglichkeiten eines steganografischen Verfahrens untersucht.\\\\
Als Ergebnis einer ersten Überlegung \eqref{sec:bild-these} hat sich gezeigt,
dass Änderungen bis zur vierten Bitstelle im Farbkanal eines Bilds nur
schwer zu erkennen sind. Es wurde hieraus geschlossen, dass ein
Algorithmus, welcher nach dem \acs{lsb}-Verfahren arbeitet,
grundsätzlich ein steganografisch gutes Ergebnis liefert.
Insbesondere kann nach \eqref{sec:bild-these} und \autoref{tab:zuordnung-n-bmax}
mit Zuversicht behauptet werden, dass diese Qualität für
ein Großteil der Bilder mit $l < \num{37.5}$ erhalten bleibt.\\\\
Es konnte anhand dieser Überlegungen schließlich ein Algorithmus vorgestellt werden
mit guter steganografischer Qualität.
Aufseiten einer Anwendung wurde gezeigt,
wie Nachrichten benutzergebunden verschlüsselt werden. In Kombination
mit dem Algorithmus konnte somit ein System beschrieben werden, welches einen
sicheren Nachrichtenaustausch über in Bildern versteckten Informationen ermöglicht.

