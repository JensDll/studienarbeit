\chapter{Das RSA-Kryptosystem} \label{cha:rsa}
Das RSA-Kryptosystem ist eines der bekanntesten und meist
verbreiteten asymmetrischen Verschlüsselungsverfahren. Es ist benannt nach
seinen drei Erfindern Ronald Rivest, Adi Shamir und Leonard Adleman, welche das Verfahren
im Jahr 1977 veröffentli\-chten \parencite[173]{BOOK:crypto}.
Am Beispiel von RSA sollen in diesem Kapitel die
Ideen der asymmetrischen Verschlüsselung
(engl. \textit{public-key-crpytography} oder \textit{asymmetric crpytography})
vorgestellt werden. Es werden die nötigen
zahlentheoretischen Begriffe eingeführt, um nachzuvollziehen
warum die RSA-Verschlüsselung funktioniert und sicher ist.

\section{Asymmetrische Verschlüsselung}
Um die Idee der asymmetrischen Verschlüsselung besser zu verstehen,
ist es hilfreich auf das Prinzip der symmetrischen Verfahren zurückzukommen
(\autoref{fig:sym-encryption-revisit}).

\begin{figure}[h]
  \centering
  \begin{tikzpicture}
    [my rectangle/.style={draw, minimum width = 2cm, minimum height = 1cm}]
    \node[my rectangle] (e) {$e$};
    \node[my rectangle, right=4cm of e] (d) {$e^{-1}$};
    \node[below=of e] {$k$} edge [->] (e);
    \node[below=of d] {$k$} edge [->] (d);

    \draw [->] (e) to node[above] {$y$} (d);

    \draw [<-] (e.west) to
    node[pos=1, above, align=center] {Alice \\\\ $x$} +(-1.5cm,0);
    \draw [->] (d.east) to
    node[pos=1, above, align=center] {Bob \\\\ $x$} +(1.5cm, 0);
  \end{tikzpicture}
  \caption{Das Prinzip der symmetrischen Verschlüsselung}
  \label{fig:sym-encryption-revisit}
\end{figure}

\noindent
Ein solches System hat zwei symmetrische Eigenschaften:
\begin{enumerate}
  \item Derselbe geheime Schlüssel wird sowohl für die
        Verschlüsselung als auch Entschlüs\-selung verwendet.
  \item Die Ver- und Entschlüsselungsfunktion sind sich sehr ähnlich
        (im Fall von \acs{otp} oder \acs{des} sind sie sogar gleich).
\end{enumerate}
Verfahren wie \acs{aes} sind sehr sicher und schnell.
Ihre symmetrischen Eigenschaften führen jedoch auch zu Problemen
und Einschränkungen,
welche im Folgenden beschrieben werden:

\paragraph{Schlüsselverteilungsproblem}
Da beide Parteien denselben Schlüssel benötigen, muss dieser in
irgendeiner Form untereinander verteilt werden.
Der direkte Austausch über die Schnittstelle ist nicht möglich, denn
die öffentliche Kommunikationsstrecke (unsicherer Kanal) ist nicht geschützt gegen Abhören.

\paragraph{Anzahl der Schlüssel}
Auch wenn das Schlüsselverteilungsproblem gelöst werden kann, entstehen
schnell Probleme, da die Anzahl der Schlüssel in einem
Netz mit zunehmender Teilnehmerzahl stark wächst. In einem Netz mit $n$
Teilnehmern, wobei jeder Teilnehmer mit jedem verschlüsselt kommunizieren soll, gibt es
\begin{equation*}
  \binom{n}{2} = \frac{n!}{2! \cdot (n-2)!} = \frac{n\cdot(n-1)}{2} \approx \frac{n^2}{2}
\end{equation*}
verschiedene Schlüsselpaare und jeder Teilnehmer muss $n-1$ Schlüssel speichern.
\begin{example}
  In einem Netz mit 500 Teilnehmern gibt es bereits $500 \cdot 499 / 2 = 124.750$ Schlüsselpaare
  und $124.750 \cdot 2 = 249.500$ Schlüssel müssen verteilt werden.
\end{example}
\noindent
Das Problem ist auch bekannt als das
$n^2$-Schlüsselverteilungsproblem \parencite[334-335]{BOOK:crypto}.

\paragraph{Verbindlichkeit}
Es ist nur mit symmetrischer Verschlüsselung nicht möglich, einer dritten Person
zu beweisen, welcher Gesprächsteilnehmer eine Nachricht erstellt hat.
Es gibt jedoch viele Bereiche, in denen dieser Beweis wichtig ist, beispielsweise
im Onlinehandel:
\begin{example}
  Alice betreibt einen Onlinehandel, sie muss beweisen können, dass ein
  Käufer Bob eine Bestellung getätigt hat,
  denn anderenfalls könnte er jederzeit behaupten,
  Alice hätte diese fälschlicherweise erstellt.
\end{example}
\noindent
Das Problem ist auch bekannt als Non-Repudiation \parencite{SITE:nonrepudiation}
und kann gelöst werden durch digitale Signaturen.\\[8pt]
Asymmetrische Verschlüsselungsverfahren bieten mögliche Lösungen zu den eben
beschriebenen Problemen, um eine Nachricht zu verschlüsseln, ist es nicht mehr nötig,
dass der Absender in Besitz von geheimer Information ist. Solch ein System wird
realisiert, indem Bob einen Schlüssel veröffentlicht,
welcher jedem Netzteilnehmer
zur Verfügung steht und frei verwendet werden kann.
Bob hat außerdem einen privaten Schlüssel, welchen nur er kennt.
Bobs Schlüssel $k$ besteht also aus einem öffentlichen Teil, $\keypub{}$ (\textit{key public}),
und einem privaten Teil, $\keypr{}$ (\textit{key private}). Wichtig ist, dass eine Nachricht,
welche mit Bobs öffentlichen Schlüssel verschlüsselt wurde, nur auch mit Bobs privaten
Schlüssel wieder entschlüsselt werden kann. Ein einfaches Protokoll,
welches nach diesem Prinzip arbeitet ist in \autoref{fig:asym-encr} zu sehen.

\begin{figure}[h]
  \centering
  \begin{tikzpicture}
    \node at (0,0) {\textbf{Alice}};
    \node at (0,-2) {$y = e_{\keypub{}}(x)$};

    \node at (8,0) {\textbf{Bob}};
    \node at (8,-1) {$(\keypub{},\keypr{}) = k$};
    \node at (8,-3) {$x = d_{\keypr{}}(y)$};

    \path [->]
    (6,-1) edge node[above] {$\keypub{}$} (2,-1)
    (2,-3) edge node[above] {$y$} (6,-3);
  \end{tikzpicture}
  \caption{Das Prinzip der asymmetrischen Verschlüsselung}
  \label{fig:asym-encr}
\end{figure}

\noindent
Ein Nachrichtenaustausch ist somit ohne sicheren Kanal möglich. Der oben beschriebene
Ablauf kann nun so modifiziert, um einen symmetrischen Schlüssel auszutauschen, beispielsweise
für \acs{aes}. Alice generiert einen symmetrischen Schlüssel und verschlüsselt ihn mit einem
asymmetrischen Verfahren. Bob kann die Nachricht entschlüsseln und ist somit ebenfalls
im Besitz des Schlüssels. Wie in \autoref{fig:asym-encr-key-exchange} zu sehen ist,
kann die restliche Kommunikation jetzt mit einem symmetrischen
Verfahren gesichert werden. Es ist durchaus wünschenswert, nicht dauerhaft asymmetrisch
zu verschlüsseln, da dies im Gegensatz zu dem symmetrischen Gegenstück sehr viel
rechenintensiver ist.
\newpage

\begin{figure}[h]
  \centering
  \begin{tikzpicture}
    \node at (0,0) {\textbf{Alice}};
    \node[align=center] at (0,-2) {
      Zufälligen Schlüssel $k$ wählen \\
      $y = e_{\keypub}(k)$
    };
    \node[align=center] at (0,-4) {
      Nachricht $x$ verschlüsseln \\
      $z = AES_k(x)$
    };

    \node at (10,0) {\textbf{Bob}};
    \node at (10,-1) {$(\keypub,\keypr)$};
    \node at (10,-3) {$k = d_{\keypr}(y)$};
    \node at (10,-5) {$x = AES_k^{-1}(z)$};

    \path [->]
    (7,-1) edge node[above] {$\keypub{}$} (3,-1)
    (3,-3) edge node[above] {$y$} (7,-3)
    (3,-5) edge node[above] {$z$} (7,-5);
  \end{tikzpicture}
  \caption{Schlüsselaustausch mit asymmetrischer Verschlüsselung}
  \label{fig:asym-encr-key-exchange}
\end{figure}

\noindent
Asymmetrische Verfahren basieren alle auf einem zugrunde liegenden Prinzip:
Der Einwegfunktion oder auch Falltürfunktion genannt. Es kann folgende
Definition gegeben werden \parencite[153]{BOOK:crypto}:

\begin{definition}[Einwegfunktion]
  Eine Funktion $f$ heißt Einwegfunktion, wenn gilt:
  \begin{enumerate}
    \item Der Funktionswert $y = f(x)$ ist komplexitätstheoretisch einfach berechenbar,
          d.\,h. die Laufzeit des Algorithmus wächst
          nicht stärker als eine Polynomfunktion (Polynomialzeit).
    \item Die Umkehrfunktion $x = f^{-1}(y)$ ist komplexitätstheoretisch schwierig berechenbar,
          d.\,h. es gibt keinen bekannten Algorithmus welcher das Problem in angemessener
          Zeit lösen kann, zum Beispiel in 1000 Jahren.
  \end{enumerate}
\end{definition}

\noindent
Es gibt zwei Einwegfunktionen, welche in der Praxis häufig eingesetzt werden.
Die erste Funktion, welche auch in RSA verwendet wird,
basiert auf dem Faktorisieren großer natürlicher Zahlen.
Es ist einfach ein Produkt zu berechnen, jedoch ist es schwierig, eine Zahl zu
faktorisieren. Die zweite Funktion basiert auf dem Lösen diskreter Logarithmen.
Das Berechnen diskreter Logarithmen ist ein rechenintensives Problem
und bildet unter anderem die Einwegfunktion des
Diffie-Hellman-Schlüsselaustausches.
Das gesamte Verfahren hat jedoch keine so intuitive Beschreibung.

\section{Zahlentheoretische Grundlagen}
Es werden in diesem Abschnitt die zahlentheoretischen Grundlagen beschrieben,
um nachzuvollziehen warum die RSA-Verschlüsselung funktioniert.

\subsection{Der Euklidische Algorithmus}
Es wird begonnen mit dem Begriff des größten gemeinsamen Teilers.
Es seien $a,b \in  \mathbb{N}$, es bezeichnet $T(a)$ die Menge aller Teiler von $a$,
dann heißt jedes $t \in T(a) \cap T(b)$ gemeinsamer Teiler von $a$ und $b$.

\begin{example}
  \begin{align*}
    T(6)           & = \{1,2,3,6\}      \\
    T(4)           & = \{1,2,4\}        \\
    T(6) \cap T(4) & = \{1,2\} \qedhere
  \end{align*}
\end{example}

\begin{definition}[Größter gemeinsamer Teiler]
  Es seien $a,b \in \mathbb{N}$.\\
  Es sei $T = T(a) \cap T(b)$.\\
  Es sei $g \in T$ und es gelte für alle $t \in T$: $t \leq g$. Dann heißt $g$
  größter gemeinsamer Teiler von $a$ und $b$.
  Man schreibt auch $g = \ggt{a}{b}$.
\end{definition}

\paragraph{Rechenregeln für ggT:}
Für $a,b \in \mathbb{N}$ gilt \parencite{SITE:euklid}:
\begin{enumerate}[ref=(\arabic*)]
  \item $\ggt{a}{a} = a$ \label{enum:ggT1}
  \item $\ggt{a}{1} = 1$ \label{enum:ggT2}
  \item $\ggt{a}{0} = a$ \label{enum:ggT3}
  \item $\ggt{a}{b} = \ggt{b}{a}$ \label{enum:ggT4}
  \item $\ggt{a}{b} = \ggt{a - b}{b}$ \label{enum:ggT5}
\end{enumerate}

\begin{proof}\mbox{}
  \begin{enumerate}
    \item Wegen $T(a) \cap T(a) = T(a)$.
    \item Wegen $T(1) = \{1\}$ und $T(a) \cap \{1\} = \{1\}$.
    \item Wegen $T(0) = \mathbb{N}$ und $T(a) \cap \mathbb{N} = T(a)$.
    \item Wegen der Kommutativität der Schnittmenge $T(a) \cap T(b) = T(b)\cap T(a)$.
    \item Zu zeigen: $\ggt{a}{b} = \ggt{a - b}{b}$ geschrieben als $g = g'$.
          \begin{center}
            Aus $g \mid a$ und $g \mid b \Rightarrow g \mid a-b \Rightarrow g \in T(a-b) \cap T(b)$.\\
            $g$ ist ein Teiler von $a-b$ und $b$.
          \end{center}
          Ist $g$ der größte gemeinsame Teiler?
          Es sei $h \in T(a-b) \cap T(b)$ und $h > g$:
          \begin{center}
            Aus $h \mid a-b$ und $h \mid b \Rightarrow h \mid a$.\\
            $h \mid a$ und $h \mid b$ und $h > g$ führt zum Wiederspruch.
          \end{center}
  \end{enumerate}
\end{proof}

\begin{example}
  Es seien $a=132$ und $b=51$, dann gilt:
  \begin{align*}
    \ggt{132}{51} & \numeq{\ref{enum:ggT5}} \ggt{132 - 51}{51} =
    \ggt{81}{51} \numeq{\ref{enum:ggT5}} \ggt{30}{51} \numeq{\ref{enum:ggT4}}
    \ggt{51}{30} \numeq{\ref{enum:ggT5}}                                          \\
    \ggt{21}{30}  & \numeq{\ref{enum:ggT4}}  \ggt{30}{21} \numeq{\ref{enum:ggT5}}
    \ggt{9}{21} \numeq{\ref{enum:ggT4}}
    \ggt{21}{9} \numeq{\ref{enum:ggT5}} \ggt{12}{9} \numeq{\ref{enum:ggT5}}       \\
    \ggt{3}{9}    & \numeq{\ref{enum:ggT4}} \ggt{9}{3} \numeq{\ref{enum:ggT5}}
    \ggt{6}{3} \numeq{\ref{enum:ggT5}} \ggt{3}{3} \numeq{\ref{enum:ggT1}} 3 \qedhere
  \end{align*}
\end{example}

\noindent
Es ist zu sehen, dass sich Rechenregel \ref{enum:ggT5} iterativ anwenden lässt:
\begin{equation*}
  \ggt{a}{b} = \ggt{a-b}{b} = \ggt{a-2b}{b} = \dots = \ggt{a-mb}{b}
\end{equation*}
Wird ein maximales $m$ gewählt, mit der Bedingung $(a-mb) > 0$, kann der Algorithmus
in minimalen Schritten durchgeführt werden. Dies ist der Fall für die Division mit Rest:
\begin{equation*}
  \ggt{a}{b} = \ggt{a \mod{b}}{b}
\end{equation*}

\begin{example}
  Es seien $a=132$ und $b=51$, dann gilt:
  \begin{align*}
    132 & = 2 \cdot 51 + 30 & \ggt{132}{51} & = \ggt{51}{30}            \\
    51  & = 1 \cdot 30 + 21 & \ggt{51}{30}  & = \ggt{30}{21}            \\
    30  & = 1 \cdot 21 + 9  & \ggt{30}{21}  & = \ggt{21}{9}             \\
    21  & = 2 \cdot 9 + 3   & \ggt{21}{9}   & = \ggt{9}{3}              \\
    9   & = 3 \cdot 3 + 0   & \ggt{9}{3}    & = \ggt{3}{0} = 3 \qedhere
  \end{align*}
\end{example}
\noindent
In allgemeiner Form können die eben gezeigten Schritte folgendermaßen beschrieben werden:
\begin{equation}
  \label{eq:euklid}
  \begin{split}
    a_i     & = q_i \cdot b_i + r_i \\
    a_{i+1} & = q_{i+1} \cdot b_{i+1} + r_{i+1} \\
    \text{mit} \quad a_{i+1} & = b_i \\
    b_{i+1} & = r_i = a_i - q_i \cdot b_i
  \end{split}
\end{equation}
Eine rekursive Implementierung des Euklidischen Algorithmus ist in \autoref{alg:ggt}
zu sehen (engl. \textit{greatest common divisor} (gcd)).
Das Verfahren terminiert nachdem das erste Mal ein Rest von null berechnet wurde.
\begin{singlespace}
  \begin{algorithm}
    \DontPrintSemicolon
    \KwIn{zwei ganze Zahlen $a$ und $b$}
    \KwOut{$\ggt{a}{b}$}
    \BlankLine
    \Fn(){$\ggt{a}{b}$}{
      $r \leftarrow a \mod{b}$\;
      \If(){$r = 0$}{
        \Return{$b$}
      }
      \Return{$\ggt{a}{b}$}
    }
    \caption{Euklidischer Algorithmus}
    \label{alg:ggt}
  \end{algorithm}
\end{singlespace}

\subsection{Der erweiterte Euklidische Algorithmus}
Es wurde gezeigt, dass der größte gemeinsame Teiler zweier Zahlen durch
das Reduzieren der Operanden ermittelt werden kann. In der Kryptografie ist
das Finden dieser Zahl allerdings nicht das Hauptanwendungsgebiet des
Algorithmus. Es stellt sich heraus, dass eine Erweiterung des Euklidischen Algorithmus
verwendet werden kann, um multiplikative Inverse Modulo $m$ zu ermitteln.
Betrachtet man den Ring $\mathbb{Z}_m$, dann ist das Inverse $a^{-1}$
einer Zahl $a \in \mathbb{Z}_m$ gegeben durch die folgende Beziehung:
\begin{equation}
  \label{eq:inverse}
  a \cdot a^{-1} \equiv 1 \pmod{m}
\end{equation}
Das multiplikative Inverse existiert nicht für alle Elemente. Es kann aber eine Aussage
darüber getroffen werden, wann es existiert. Ein Element
$a \in \mathbb{Z}_m$ besitzt genau dann ein Inverses, wenn gilt $\ggt{a}{m} = 1$.
Zwei Zahlen $a$ und $b$ für die gilt $\ggt{a}{b} = 1$ nennt man teilerfremd oder
relativ Prim (Symbol $a \perp b$, engl. \textit{relatively prime} oder \textit{coprime}).
Der Erweiterte Euklidische Algorithmus berechnet eine Linearkombination der folgenden Form,
welche auch als diophantische Gleichung
bezeichnet wird \parencite[160]{BOOK:crypto} \parencite{SITE:diophant}:
\begin{equation}
  x \cdot a + y \cdot b = \ggt{a}{b}
\end{equation}

\begin{example}
  Es seien erneut $a=132$ und $b=51$, es wird zuerst der $\ggt{132}{51}$ mit dem
  Euklidischen Algorithmus bestimmt:
  \begin{align*}
    132 & = 2 \cdot 51 + 30 \\
    51  & = 1 \cdot 30 + 21 \\
    30  & = 1 \cdot 21 + 9  \\
    21  & = 2 \cdot 9 + 3   \\
    9   & = 3 \cdot 3 + 0
  \end{align*}
  Der $\ggt{132}{51}$ ist mit 3 bestimmt. Es soll nun versucht werden, ein $x$ und $y$
  zu finden, sodass $3 = x \cdot 132 + y \cdot 51$. Aus der vorletzten Zeile weiß man:
  \begin{equation*}
    3 = 21 - 2 \cdot 9
  \end{equation*}
  Wie kam die 9 zustande? Aus der Zeile darüber mit $9 = 30 - 21$.
  Eingesetzt und zusammengefasst:
  \begin{equation*}
    3 = 21 - 2 \cdot (30 - 21) = 21 - 2 \cdot 30 + 2 \cdot 21 = -2 \cdot 30 + 3 \cdot 21
  \end{equation*}
  Wie kam die 21 zustande? Erneut aus der Zeile darüber mit $21 = 51 - 30$.
  Eingesetzt, zusammengefasst und für die letzten beiden Zeilen fortgeführt:
  \begin{align*}
    3 & = -2 \cdot 30 + 3 \cdot (51 - 30) =
    -2 \cdot 30 + 3 \cdot 51 - 3 \cdot 30 = 3 \cdot 51 - 5 \cdot 30                          \\
    3 & = 3 \cdot 51 - 5 \cdot (132 - 2 \cdot 51) = 3 \cdot 51 - 5 \cdot 132 + 10 \cdot 51 =
    -5 \cdot 132 + 13 \cdot 51
  \end{align*}
  Man erhält die gewünschte Gleichung.
\end{example}

\noindent
Das Verfahren kann erneut allgemein betrachtet werden. Durch Rückwärtsarbeiten erhält
man verschiedene Gleichungen der folgenden Form:
\begin{align*}
  \ggt{a}{b} & = x_i \cdot a_i + y_i \cdot b_i                 \\
  \ggt{a}{b} & = x_{i+1} \cdot a_{i+1} + y_{i+1} \cdot b_{i+1}
\end{align*}
Außerdem kennen wir die Gleichungen aus \eqref{eq:euklid}:
\begin{align*}
  a_i                      & = q_i \cdot b_i + r_i             \\
  a_{i+1}                  & = q_{i+1} \cdot b_{i+1} + r_{i+1} \\
  \text{mit} \quad a_{i+1} & = b_i                             \\
  b_{i+1}                  & = r_i = a_i - q_i \cdot b_i
\end{align*}
Einsetzen und umformen:
\begin{align*}
  \ggt{a}{b} & = x_i \cdot a_i + y_i \cdot b_i                               \\
             & = x_i \cdot (q_i \cdot b_i + r_i) + y_i \cdot a_{i+1}         \\
             & = x_i \cdot (q_i \cdot a_{i+1} + b_{i+1}) + y_i \cdot a_{i+1} \\
             & = x_i \cdot b_{i+1} + (y_i + x_i \cdot q_i) \cdot a_{i+1}
\end{align*}
Es ergeben sich die folgenden Regeln:
\begin{align}
  x_i & = y_{i+1} \\
  \begin{split}
    y_i + x_i \cdot q_i &= x_{i+1} \\
    y_i &= x_{i+1} - x_i \cdot q_i \\
    y_i &= x_{i+1} - y_{i+1} \cdot q_i
  \end{split}
\end{align}
Dies resultiert im erweiterten Euklidischen Algorithmus, welcher in Tabellenform
einfach und schnell durchführbar ist:

\begin{table}[h]
  \caption{Erweiterter Euklidischer Algorithmus}
  \centering
  \renewcommand{\arraystretch}{1.4}
  \begin{tabular}{|l|l|l|l|l|l|l|l|}
    \hline
    $i$ & $a$ & $b$ & $q$ & $r$ & $x$                & $y$                                                              & Kontrolle                        \\ \hline
    1   & 132 & 51  & 2   & 30  & \hl{orange}{-5}    & $\hl{Magenta}{3} + \hl{orange}{5} \cdot 2 = 13$                  & $3 = -5 \cdot 132 + 13 \cdot 51$ \\ \hline
    2   & 51  & 30  & 1   & 21  & \hl{Magenta}{3}    & $\hl{RoyalBlue}{-2} - \hl{Magenta}{3} \cdot 1 = \hl{orange}{-5}$ & $3 = 3 \cdot 51 - 5 \cdot 30$    \\ \hline
    3   & 30  & 21  & 1   & 9   & \hl{RoyalBlue}{-2} & $\hl{Green}{1} + \hl{RoyalBlue}{2} \cdot 1 = \hl{Magenta}{3} $   & $3 = -2 \cdot 30 + 3 \cdot 21$   \\ \hline
    4   & 21  & 9   & 2   & 3   & \hl{Green}{1}      & $\hl{Red}{0} - \hl{Green}{1}  \cdot 2 = \hl{RoyalBlue}{-2}$      & $3 = 1 \cdot 21 - 2 \cdot 9$     \\ \hline
    5   & 9   & 3   & 3   & 0   & \hl{Red}{0}        & \hl{Green}{1}                                                    & $3 = 0 \cdot 9 + 1 \cdot 3$      \\ \hline
  \end{tabular}
\end{table}

\noindent
Es soll nun gezeigt werden, wie der erweiterte Euklidischen Algorithmus verwendet werden kann,
um multiplikative Inverse zu berechnen. Es soll das Inverse $a \mod{m}$ bestimmt werden, wobei
$m > a$. Das Inverse existiert genau dann, wenn $\ggt{m}{a} = 1$, dies bedeutet es gibt
eine Gleichung der Form $x \cdot m + y \cdot a = 1$. Stellt man diese Gleichung
als Kongruenzrelation Modulo $m$ dar, erhalten wir:
\begin{align*}
  x \cdot m + y \cdot a & = 1               \\
  x \cdot m + y \cdot a & \equiv 1 \pmod{m} \\
  x \cdot 0 + y \cdot a & \equiv 1 \pmod{m} \\
  a \cdot y             & \equiv 1 \pmod{m}
\end{align*}
Die letzte Zeile ist genau die Definition des Inversen \eqref{eq:inverse}, welches
mit $y$ bestimmt wurde.
\begin{example}
  Es soll das Inverse $21^{-1} \pmod{89}$ bestimmt werden. Die Zahlen 21 und 89 sind
  teilerfremd, d.\,h. das Inverse existiert und es gilt
  $\ggt{89}{21} = 1 = x \cdot 89 + y \cdot 21$. Der Euklidische
  Algorithmus kann tabellarisch durchgeführt werden:
  \begin{center}
    \centering
    \begin{tabular}{|l|l|l|l|l|l|l|}
      \hline
      $a$ & $b$ & $q$ & $r$ & $x$ & $y$                  \\ \hline
      89  & 21  & 4   & 5   & -4  & $1 + 4 \cdot 4 = 17$ \\ \hline
      21  & 5   & 4   & 1   & 1   & $0 - 1 \cdot 4 = -4$ \\ \hline
      5   & 1   & 5   & 0   & 0   & 1                    \\ \hline
    \end{tabular}
  \end{center}
  \noindent
  Wir erhalten den größten gemeinsamen Teiler als Linearkombination:
  \begin{equation*}
    -4 \cdot 89 + 17 \cdot 21 = 1
  \end{equation*}
  Es folgt hieraus: Das Inverse von 21 in $\mathbb{Z}_{89}$ beträgt 17. Dieses Ergebnis kann durch nachrechnen
  verifiziert werden:
  \begin{equation*}
    17 \cdot 21 = 357 \equiv 1 \pmod{89} \qedhere
  \end{equation*}
\end{example}

\noindent
Eine rekursive Implementierung des erweiterten Euklidischen Algorithmus ist in
\autoref{alg:ggt-e} zu sehen, das Symbol $\div$ bezeichnet die Ganzzahldivision
zweier Zahlen $a$ und $b$.
Wie zuvor terminiert das Verfahren nachdem das erste
Mal ein Rest von null berechnet wurde.
\begin{singlespace}
  \begin{algorithm}
    \DontPrintSemicolon
    \KwIn{zwei ganze Zahlen $a$ und $b$}
    \KwOut{$\ggt{a}{b}$ als auch $x$ und $y$, sodass $\ggt{a}{b} = x \cdot a + y \cdot b$}
    \BlankLine
    \Fn(){$ggT(a,b)$}{
      $r \leftarrow a \mod{b}$\;
      $q \leftarrow a \div b$\;
      \If(){$r = 0$}{
        \Return{$(0,1,b)$}
      }
      $(x,y,ggt) \leftarrow ggT(b,r)$\;

      \Return{$(y,x - y \cdot q, ggt)$}
    }
    \caption{Erweiterter Euklidischer Algorithmus}
    \label{alg:ggt-e}
  \end{algorithm}
\end{singlespace}


\subsection{Die Eulersche Phi-Funktion}
Die Eulersche Phi-Funktion befasst sich mit dem auf den ersten Blick seltsamen
Problem, die Anzahl an teilerfremden Zahlen in einer Menge zu finden. Die Regeln
und Sätze welche von der zahlentheoretischen Funktion abgeleitet werden können, sind jedoch sehr
hilfreich in der asymmetrischen Verschlüsselung und insbesondere für das
RSA-Verfahren. Die Eulersche Phi-Funktion ist folgendermaßen definiert
\parencite[165]{BOOK:crypto} \parencite{SITE:phi-euler-fermat}:

\begin{definition}[Eulersche Phi-Funktion]
  Die Anzahl der zu $m \in \mathbb{N}^\times$ teilerfremden Zahlen aus $\{1,2,\dots,m\}$,
  ist gekennzeichnet als $\phif{m}$ und heißt Eulersche Phi-Funktion. $\phif{m}$
  kann formal beschrieben werden als die Mächtigkeit der Menge:
  \begin{equation*}
    \phif{m} = \vert \{x \in \mathbb{N} \mid 1 \leq x \leq m \wedge \ggt{m}{x} = 1 \} \vert
  \end{equation*}
\end{definition}

\begin{example}
  \begin{align*}
    \phif{5}  & = \vert \{1,2,3,4\} \vert = 4                      \\
    \phif{6}  & = \vert \{1,5\} \vert = 2                          \\
    \phif{20} & = \vert \{1,3,7,9,11,13,17,19\} \vert = 8 \qedhere
  \end{align*}
\end{example}

\noindent
Die naive Berechnung der Eulerschen Phi-Funktion, alle Zahlen zu durchlaufen und den
größten gemeinsamen Teiler zu bestimmen, ist sehr langsam und für große Zahlen, wie sie
in der asymmetrischen Verschlüsselung verwendet werden, nicht möglich. Es existiert jedoch
eine Beziehung, mit welcher $\phif{m}$ bestimmt werden kann, unter der
Bedingung, dass die Primfaktorzerlegung von $m$ bekannt ist. Diese lässt sich aus
den folgenden Sätzen ableiten, welche hier im einzelnen nicht noch einmal bewiesen werden
sollen \parencite{SITE:phi-euler-fermat}:
\begin{enumerate}[ref=(\arabic*)]
  \item Es sei $p \in \mathbb{P}$, dann gilt: $\phif{p} = p - 1$. \label{enum:phi1}
  \item Es sei $p \in \mathbb{P}$, dann gilt: $\phif{p^n} = p^n - p^{n-1}$. \label{enum:phi2}
  \item Es sei $\ggt{n}{m} = 1$, dann gilt:
        $\phif{n \cdot m} = \phif{n} \cdot \phif{m}$. \label{enum:phi3}
\end{enumerate}

\begin{satz}[Formel zur Eulerschen Phi-Funktion]
  Es sei
  \begin{equation*}
    n = p_1^{m_1} \cdot p_2^{m_2} \cdot \ldots \cdot p_r^{m_r} = \prod_{i=1}^{r} p_i^{m_i}
  \end{equation*}
  die Primfaktorzerlegung einer natürlichen Zahl mit $p_i$ unterschiedlichen Primzahlen
  und Exponenten $m_1 \geq 1,\ldots, m_r \geq 1$. Dann ist:
  \begin{equation*}
    \phif{n} = \prod_{i=1}^{r} p_i^{m_i} - p_i^{m_i - 1} =
    n \cdot \prod_{i=1}^{r} 1 - \frac{1}{p_i}
  \end{equation*}
\end{satz}

\begin{proof}
  Da die Primfaktoren einer Zahl teilerfremd sind, kann nach Regel \ref{enum:phi3}
  und \ref{enum:phi2} geschrieben werden:
  \begin{align*}
    \phif{n} \numeq{\ref{enum:phi3}} \prod_{i=1}^{r} \phif{p_i^{m_i}}
    \numeq{\ref{enum:phi2}} \prod_{i=1}^{r} p_i^{m_i} - p_i^{m_i - 1}
     & = \prod_{i=1}^{r} p_i^{m_i} \cdot (1 - \frac{1}{p_i})               \\
     & = \prod_{i=1}^{r} p_i^{m_i} \cdot \prod_{i=1}^{r} 1 - \frac{1}{p_i} \\
     & = n \cdot \prod_{i=1}^{r} 1 - \frac{1}{p_i} \qedhere
  \end{align*}
\end{proof}

\begin{example}
  \begin{align*}
    \phif{6 = 2 \cdot 3}        & = 6 \cdot (1 - \frac{1}{2}) \cdot (1 - \frac{1}{3}) =
    6 \cdot \frac{2}{6} = 2                                                              \\
    \phif{20 = 2^2 \cdot 5}     & = 20 \cdot (1 - \frac{1}{2}) \cdot (1 - \frac{1}{5}) =
    20 \cdot \frac{4}{10} = 8                                                            \\
    \phif{1000 = 2^3 \cdot 5^3} & = 1000 \cdot \frac{4}{10} = 400 \qedhere
  \end{align*}
\end{example}

\subsection{Satz von Euler und Fermat}
Es werden im Folgenden zwei Sätze vorgestellt von Euler und Fermat, die
in der Zahlentheorie und vielen allen asymmetrischen Verschlüsselungsverfahren
eine sehr wesentliche Bedeutung haben.
Es muss zuvor jedoch noch eine Eigenschaft der Restklassen festgehalten werden:

\begin{satz}[Eindeutigkeit des $ggT$ von Repräsentanten einer Restklasse und dem Modul $m$]
  \label{satz:restklasse-ggt}
  Für einen Modul $m$ und zwei Repräsentanten $a,b \in \mathbb{N}$ derselben
  Restklasse $\vert a \vert_m = \vert b \vert_m$ gilt:
  \begin{equation*}
    \ggt{a}{m} = \ggt{b}{m}
  \end{equation*}
\end{satz}
\begin{proof}
  Es ist $b = a + tm$ mit $t \in \mathbb{Z}$ nach Definition gegeben. Es sind
  \begin{align*}
    g_1 & = \ggt{a}{m} = x \cdot \hl{Red}{a} + y \cdot \hl{Red}{m}     \\
    g_2 & = \ggt{b}{m} = p \cdot \hl{Green}{b} + q \cdot \hl{Green}{m}
  \end{align*}
  mit $x,y,p,q \in \mathbb{Z}$. Es gilt außerdem allgemeiner: Sei $c = x \cdot a + y \cdot m$,
  dann ist $c$ ein Vielfaches von $\ggt{a}{m}$.
  Man schreibt:
  \begin{align*}
    g_1 & = xa + ym                                    & g_2 & = pb + qm                                \\
        & = x(b - tm) + ym                             &     & = p(a + tm) + qm                         \\
        & = xb - xtm + ym                              &     & = pa + ptm + qm                          \\
        & = x\,\hl{Green}{b} - (xt + y)\,\hl{Green}{m} &     & = p\,\hl{Red}{a} + (pt + q)\,\hl{Red}{m} \\
        & \Rightarrow g_1 \mid g_2                     &     & \Rightarrow g_2 \mid g_1
  \end{align*}
  Aus den beiden letzten Zeile folgt: $g_1 = g_2$.
\end{proof}


\begin{satz}[Der Satz von Euler]
  Es seien $a,m \in \mathbb{N}^\times$ mit $a \perp m$, dann gilt:
  \begin{equation*}
    a^{\phif{m}} \equiv 1 \pod{m}
  \end{equation*}
\end{satz}
\begin{proof}
  Mit gleichem Grundgedanken in \parencite{SITE:phi-euler-fermat} und
  \parencite[187-188]{BOOK:numberTheory}. \\
  Es ist $n := \phif{m}$. Es seien $x_1,x_2,\ldots,x_n$ die $n$ verschiedenen zu
  $m$ teilerfremden Zahlen aus der Menge $\{1,2,\ldots,m\}$.
  Mit einer vorgegeben Zahl
  $a \in \mathbb{N}^\times \wedge a \perp m$ bilden wir die Produkte $ax_1,ax_2,\ldots,ax_n$.
  Wir betrachten ein Beispiel um eine Vermutung zu bestätigen:
  \begin{example}
    Es ist $\phif{8} = |\{1,3,5,7\}| = 4$. Wir wählen $a = 3$ und bilden in $\mathbb{Z}_8$ die Produkte:
    $3\cdot \{1,3,5,7\} = \{3,9,15,21\} \equiv \{3,1,7,5\} \pod{8}$. Die Multiplikation von $a$ erzeugt eine
    Permutation der Menge $\{1,3,5,7\}$.
  \end{example}
  \noindent
  Es sind $|x_1|_m,\ldots,|x_n|_m$ genau alle zu $m$ teilerfremden Restklassen.
  \footnote{
    Nach \autoref{satz:restklasse-ggt} wissen wir, ist eine Zahl einer Restklasse teilerfremd zu $m$,
    gilt dies auch für alle anderen Zahlen der Restklasse.
  }
  Es gilt $x_i \perp m$ und $a \perp m$, wonach auch das Produkt $ax_i$ teilerfremd zu $m$ sein muss.\\
  Es folgt $ax_i \bmod{m} = x_j$ und wir halten fest:
  \begin{equation*}
    \label{eq:proof-satz-euler}
    \forall ax_i \, \exists x_j: ax_i \equiv x_j \pod{m} \tag{$*$}
  \end{equation*}
  Es bleibt zu zeigen, dass in \eqref{eq:proof-satz-euler} keine Doppelungen entstehen,
  also alle $ax_i$ Teil unterschiedlicher Restklassen sind. Es muss gelten:
  \begin{equation*}
    ax_i \not\equiv ax_j \pod{m} \qquad i \neq j
  \end{equation*}
  Angenommen $ax_i \equiv ax_j \pod{m}$ für $i \neq j$, dann gilt nach der Kürzungsregel
  \footnote{
    Da das Inverse wegen $a \perp m$ existiert, kann mit diesem multipliziert werden und die $a$'s
    fallen weg.
  }
  $x_i \equiv x_j \pod{m}$, was nicht sein kann.
  Wegen $x_1,x_2,\ldots,x_n \perp m$ liefert die Kürzungsregel, wenn man noch
  $n = \phif{m}$ beachtet, die ursprüngliche Behauptung:
  \begin{align*}
    ax_1 \cdot ax_2 \cdot \ldots \cdot ax_n & \equiv
    x_1 \cdot x_2 \cdot \ldots \cdot x_n  \pod{m}                       \\
    a^{\phif{m}}                            & \equiv 1 \pod{m} \qedhere
  \end{align*}
\end{proof}

\noindent
Der Kleine Satz von Fermat kann jetzt als Spezialform des Satzes von Euler (mit $m = p$) direkt
aufgeschrieben werden:

\begin{satz}[Kleiner Satz von Fermat]
  \label{satz:kleiner-fermat}
  Es sei $a \in \mathbb{N}^\times$ und $p \in \mathbb{P}$ mit $a \perp p$, dann gilt:
  \begin{equation*}
    a^{p-1} \equiv 1 \pod{p}
  \end{equation*}
\end{satz}

\noindent
Der Kleine Satz von Fermat wird in der Kryptografie unter anderem dafür verwendet,
Primzahltests durchzuführen.

\section{Das RSA-Verfahren}
Das RSA-Verfahren basiert auf der Annahme, dass es nicht effizient
möglich ist, die Primfaktorzerlegung einer natürlichen Zahl zu bestimmen.
Die Generierung des privaten und öffentlichen Schlüssels kann in die
folgenden fünf Schritte unterteilt werden \parencite[176]{BOOK:crypto}:
\newpage
\begin{definition}[RSA Schlüsselgenerierung]\mbox{}
  \begin{enumerate}
    \item Wähle zwei sehr große Primzahlen $p$ und $q$, beispielsweise mit einer Länge
          von 512 oder 1024 Bit.
    \item Berechne $N = p \cdot q$.
    \item Berechne $\phif{N} = \phif{p \cdot q} = \phif{p} \cdot \phif{q} = (p-1) \cdot (q-1)$.
    \item Wähle eine Zahl $e$ mit $1 \leq e < \phif{N}$ und $\ggt{e}{\phif{N}} = 1$. Jetzt bilden
          $(N,e)$ den öffentlichen Schlüssel.
    \item Berechne $d$ mit $e \cdot d \equiv 1 \pmod{\phif{N}}$ und dem erweiterten
          Euklidischen Algorithmus. Das Inverse existiert, denn es gilt $e \perp \phif{N}$.
          Jetzt bilden $(N,d)$ den privaten Schlüssel.
  \end{enumerate}
\end{definition}

\noindent
Sind $p$ und $q$ entsprechend groß, ist es nicht in annehmbarer Zeit möglich, die Faktorisierung
von $N$ zu bestimmen und damit auch $\phif{N}$. Die Ver- und Entschlüsselung mit RSA
gestaltet sich sehr einfach:

\begin{definition}[Das RSA-Verfahren]
  Es seien der öffentliche und private Schlüssel gegeben mit
  $\keypub = (N,e)$ und $\keypr = (N,d)$. Dann gilt
  \begin{description}
    \item[Verschlüsselung:] $y = e_{\keypub}(x) \equiv x^e \pmod{N}$
    \item[Entschlüsselung:] $x = d_{\keypr}(y) \equiv y^d \pmod{N}$
  \end{description}
  mit $x,y \in \mathbb{Z}_N$.
\end{definition}

\noindent
Bevor die Korrektheit von RSA bewiesen wird, schreiben wir noch einen Satz auf, welcher im
Beweis verwendet werden kann:

\begin{satz}
  \label{satz:composite-mod}
  $m$ ist eine zusammengesetzte Zahl der Form $m = k_1 \cdot k_2 \cdot \ldots \cdot k_n$ und
  es gilt $a \equiv b \pmod{m}$, dann ist auch:
  \begin{equation*}
    a \equiv b \pmod{k_1} \quad\ldots\quad a \equiv b \pmod{k_n}
  \end{equation*}
\end{satz}
\begin{proof}
  Die eben gezeigten Kongruenzen lassen sich als Gleichungen schreiben:
  \begin{align*}
    a - b & = t_0 \cdot m   \\
    a - b & = t_1 \cdot k_1 \\
    \vdots                  \\
    a - b & = t_n \cdot k_n
  \end{align*}
  Wobei $t_i$ irgendwelche ganzen Zahlen sind. Es lässt sich schreiben:
  \begin{equation*}
    t_0 \cdot m = t_1 \cdot k_1 = \ldots = t_n \cdot k_n
  \end{equation*}
  Man wählt jetzt
  \begin{equation*}
    t_1 = \frac{t_0m}{k_1} \quad\ldots\quad t_n = \frac{t_0m}{k_n}
  \end{equation*}
  und findet immer eine Lösung.
\end{proof}

\noindent
Es soll nun Bewiesen werden, warum das RSA-Verfahren funktioniert:

\begin{proof}
  Die Entschlüsselung des Geheimtextes muss erneut den Klartext ergeben.
  \begin{equation}
    \label{eq:proof-rsa-3.5}
    d_{\keypr}(e_{\keypub}(x)) = x \equiv x^{ed} \pmod{N} \tag{$*$}
  \end{equation}
  Wir wissen per Definition:
  \begin{align}
    e \cdot d & \equiv 1 \pmod{\phif{N}} \notag                           \\
    \label{eq:proof-rsa-3.6}
    e \cdot d & = t \cdot \phif{N} + 1 \qquad t \in \mathbb{Z} \tag{$**$}
  \end{align}
  Um $x \equiv x^{ed} \pmod{pq}$ zu zeigen, reicht es nach \eqref{satz:composite-mod}
  die Faktoren einzeln zu betrachten.
  \begin{enumerate}
    \item Um $x \equiv x^{ed} \pmod{p}$ zu beweisen, betrachten wir zwei Fälle:
          \begin{enumerate}[topsep=0pt]
            \item Es gilt $p \mid x$ und es folgt:
                  $x \equiv 0 \pmod{p}$ woraus folgt $x^{ed} \equiv 0 \pmod{p}$
            \item Es gilt $p \nmid x \Rightarrow p \perp x$, einsetzen von \eqref{eq:proof-rsa-3.6} in
                  \eqref{eq:proof-rsa-3.5}
                  und umformen:
                  \begin{equation*}
                    x^{ed} \equiv x^{t\phif{N} + 1} \equiv
                    x^{t\phif{N}} \cdot x \equiv
                    x^{t(p - 1)(q - 1)} \cdot x \equiv
                    (x^{p - 1})^{t(q - 1)} \cdot x \pmod{p}
                  \end{equation*}
                  Nach dem Kleinen Satz von Fermat ist jetzt zu sehen:
                  \begin{equation*}
                    x \equiv (x^{p - 1})^{t(q - 1)} \cdot x \equiv
                    1^{t(q - 1)} \cdot x \equiv x \pmod{p}
                  \end{equation*}
          \end{enumerate}
    \item Die Überlegung für $q$ kann analog durchgeführt werden. Es müssen hierfür
          alle $p$ und $q$ vertauscht werden. \qedhere
  \end{enumerate}
\end{proof}

\begin{example}\mbox{}
  \begin{enumerate}
    \item Wähle $p = 3$ und $q = 11$.
    \item Berechne $N = 33$.
    \item Berechne $\phif{33} = (3 - 1) \cdot (11 - 1) = 20$.
    \item Wähle $e = 3$.
    \item Berechne $d$ mit dem erweiterten Euklidischen Algorithmus:
          \begin{center}
            \begin{tabular}{|l|l|l|l|l|l|}
              \hline
              $a$ & $b$ & $q$ & $r$ & $x$ & $y$                     \\ \hline
              20  & 3   & 6   & 2   & -1  & $1 + 1 \cdot 6 = 7 = d$ \\ \hline
              3   & 2   & 1   & 1   & 1   & $0 - 1 \cdot 1 = -1$    \\ \hline
              2   & 1   & 2   & 0   & 0   & 1                       \\ \hline
            \end{tabular}
          \end{center}
          Das Ergebnis stimmt, denn es gilt $e \cdot d = 3 \cdot 7 = 21 \equiv 1 \pmod{20}$.
    \item Alice kennt den öffentlichen Schlüssel $\keypub = (N,e) = (33,3)$ und kann eine
          Nachricht $x = 4$ verschlüsseln mit $y = x^e = 4^3 \equiv 31 \pmod{33}$
    \item Bob kennt den privaten Schlüssel $\keypr = (N,d) = (33,7)$ und kann die Nachricht
          entschlüsseln mit $y^d = 31^7 \equiv (-2)^7 \equiv (-2)^5 \cdot 2^2 \equiv
            1 \cdot 4 \equiv 4 = x \pmod{33}$. \qedhere
  \end{enumerate}
\end{example}