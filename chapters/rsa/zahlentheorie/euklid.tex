\subsection{Der Euklidische Algorithmus}
Es wird begonnen mit dem Begriff des größten gemeinsamen Teilers.
Es seien $a,b \in  \mathbb{N}$, es bezeichnet $T(a)$ die Menge aller Teiler von $a$,
dann heißt jedes $t \in T(a) \cap T(b)$ gemeinsamer Teiler von $a$ und $b$.

\begin{example}
  \begin{align*}
    T(6)           & = \{1,2,3,6\}      \\
    T(4)           & = \{1,2,4\}        \\
    T(6) \cap T(4) & = \{1,2\} \qedhere
  \end{align*}
\end{example}

\begin{definition}[Größter gemeinsamer Teiler]
  Es seien $a,b \in \mathbb{N}$.\\
  Es sei $T = T(a) \cap T(b)$.\\
  Es sei $g \in T$ und es gelte für alle $t \in T$: $t \leq g$. Dann heißt $g$
  größter gemeinsamer Teiler von $a$ und $b$.
  Man schreibt auch $g = \ggt{a}{b}$.
\end{definition}

\paragraph{Rechenregeln für ggT:}
Für $a,b \in \mathbb{N}$ gilt \parencite{SITE:euklid}:
\begin{enumerate}[ref=(\arabic*)]
  \item $\ggt{a}{a} = a$ \label{enum:ggT1}
  \item $\ggt{a}{1} = 1$ \label{enum:ggT2}
  \item $\ggt{a}{0} = a$ \label{enum:ggT3}
  \item $\ggt{a}{b} = \ggt{b}{a}$ \label{enum:ggT4}
  \item $\ggt{a}{b} = \ggt{a - b}{b}$ \label{enum:ggT5}
\end{enumerate}

\begin{proof}\mbox{}
  \begin{enumerate}
    \item Wegen $T(a) \cap T(a) = T(a)$.
    \item Wegen $T(1) = \{1\}$ und $T(a) \cap \{1\} = \{1\}$.
    \item Wegen $T(0) = \mathbb{N}$ und $T(a) \cap \mathbb{N} = T(a)$.
    \item Wegen der Kommutativität der Schnittmenge $T(a) \cap T(b) = T(b)\cap T(a)$.
    \item Zu zeigen: $\ggt{a}{b} = \ggt{a - b}{b}$ geschrieben als $g = g'$.
          \begin{center}
            Aus $g \mid a$ und $g \mid b \Rightarrow g \mid a-b \Rightarrow g \in T(a-b) \cap T(b)$.\\
            $g$ ist ein Teiler von $a-b$ und $b$.
          \end{center}
          Ist $g$ der größte gemeinsame Teiler?
          Es sei $h \in T(a-b) \cap T(b)$ und $h > g$:
          \begin{center}
            Aus $h \mid a-b$ und $h \mid b \Rightarrow h \mid a$.\\
            $h \mid a$ und $h \mid b$ und $h > g$ führt zum Wiederspruch.
          \end{center}
  \end{enumerate}
\end{proof}

\begin{example}
  Es seien $a=132$ und $b=51$, dann gilt:
  \begin{align*}
    \ggt{132}{51} & \numeq{\ref{enum:ggT5}} \ggt{132 - 51}{51} =
    \ggt{81}{51} \numeq{\ref{enum:ggT5}} \ggt{30}{51} \numeq{\ref{enum:ggT4}}
    \ggt{51}{30} \numeq{\ref{enum:ggT5}}                                          \\
    \ggt{21}{30}  & \numeq{\ref{enum:ggT4}}  \ggt{30}{21} \numeq{\ref{enum:ggT5}}
    \ggt{9}{21} \numeq{\ref{enum:ggT4}}
    \ggt{21}{9} \numeq{\ref{enum:ggT5}} \ggt{12}{9} \numeq{\ref{enum:ggT5}}       \\
    \ggt{3}{9}    & \numeq{\ref{enum:ggT4}} \ggt{9}{3} \numeq{\ref{enum:ggT5}}
    \ggt{6}{3} \numeq{\ref{enum:ggT5}} \ggt{3}{3} \numeq{\ref{enum:ggT1}} 3 \qedhere
  \end{align*}
\end{example}

\noindent
Es ist zu sehen, dass sich Rechenregel \ref{enum:ggT5} iterativ anwenden lässt:
\begin{equation*}
  \ggt{a}{b} = \ggt{a-b}{b} = \ggt{a-2b}{b} = \dots = \ggt{a-mb}{b}
\end{equation*}
Wird ein maximales $m$ gewählt, mit der Bedingung $(a-mb) > 0$, kann der Algorithmus
in minimalen Schritten durchgeführt werden. Dies ist der Fall für die Division mit Rest:
\begin{equation*}
  \ggt{a}{b} = \ggt{a \mod{b}}{b}
\end{equation*}

\begin{example}
  Es seien $a=132$ und $b=51$, dann gilt:
  \begin{align*}
    132 & = 2 \cdot 51 + 30 & \ggt{132}{51} & = \ggt{51}{30}            \\
    51  & = 1 \cdot 30 + 21 & \ggt{51}{30}  & = \ggt{30}{21}            \\
    30  & = 1 \cdot 21 + 9  & \ggt{30}{21}  & = \ggt{21}{9}             \\
    21  & = 2 \cdot 9 + 3   & \ggt{21}{9}   & = \ggt{9}{3}              \\
    9   & = 3 \cdot 3 + 0   & \ggt{9}{3}    & = \ggt{3}{0} = 3 \qedhere
  \end{align*}
\end{example}
\noindent
In allgemeiner Form können die eben gezeigten Schritte folgendermaßen beschrieben werden:
\begin{equation}
  \label{eq:euklid}
  \begin{split}
    a_i     & = q_i \cdot b_i + r_i \\
    a_{i+1} & = q_{i+1} \cdot b_{i+1} + r_{i+1} \\
    \text{mit} \quad a_{i+1} & = b_i \\
    b_{i+1} & = r_i = a_i - q_i \cdot b_i
  \end{split}
\end{equation}
Eine rekursive Implementierung des Euklidischen Algorithmus ist in \autoref{alg:ggt}
zu sehen (engl. \textit{greatest common divisor} (gcd)).
Das Verfahren terminiert nachdem das erste Mal ein Rest von null berechnet wurde.
\begin{singlespace}
  \begin{algorithm}
    \DontPrintSemicolon
    \KwIn{zwei ganze Zahlen $a$ und $b$}
    \KwOut{$\ggt{a}{b}$}
    \BlankLine
    \Fn(){$\ggt{a}{b}$}{
      $r \leftarrow a \mod{b}$\;
      \If(){$r = 0$}{
        \Return{$b$}
      }
      \Return{$\ggt{a}{b}$}
    }
    \caption{Euklidischer Algorithmus}
    \label{alg:ggt}
  \end{algorithm}
\end{singlespace}

\subsection{Der erweiterte Euklidische Algorithmus}
Es wurde gezeigt, dass der größte gemeinsame Teiler zweier Zahlen durch
das Reduzieren der Operanden ermittelt werden kann. In der Kryptografie ist
das Finden dieser Zahl allerdings nicht das Hauptanwendungsgebiet des
Algorithmus. Es stellt sich heraus, dass eine Erweiterung des Euklidischen Algorithmus
verwendet werden kann, um multiplikative Inverse Modulo $m$ zu ermitteln.
Betrachtet man den Ring $\mathbb{Z}_m$, dann ist das Inverse $a^{-1}$
einer Zahl $a \in \mathbb{Z}_m$ gegeben durch die folgende Beziehung:
\begin{equation}
  \label{eq:inverse}
  a \cdot a^{-1} \equiv 1 \pmod{m}
\end{equation}
Das multiplikative Inverse existiert nicht für alle Elemente. Es kann aber eine Aussage
darüber getroffen werden, wann es existiert. Ein Element
$a \in \mathbb{Z}_m$ besitzt genau dann ein Inverses, wenn gilt $\ggt{a}{m} = 1$.
Zwei Zahlen $a$ und $b$ für die gilt $\ggt{a}{b} = 1$ nennt man teilerfremd oder
relativ Prim (Symbol $a \perp b$, engl. \textit{relatively prime} oder \textit{coprime}).
Der Erweiterte Euklidische Algorithmus berechnet eine Linearkombination der folgenden Form,
welche auch als diophantische Gleichung
bezeichnet wird \parencite[160]{BOOK:crypto} \parencite{SITE:diophant}:
\begin{equation}
  x \cdot a + y \cdot b = \ggt{a}{b}
\end{equation}

\begin{example}
  Es seien erneut $a=132$ und $b=51$, es wird zuerst der $\ggt{132}{51}$ mit dem
  Euklidischen Algorithmus bestimmt:
  \begin{align*}
    132 & = 2 \cdot 51 + 30 \\
    51  & = 1 \cdot 30 + 21 \\
    30  & = 1 \cdot 21 + 9  \\
    21  & = 2 \cdot 9 + 3   \\
    9   & = 3 \cdot 3 + 0
  \end{align*}
  Der $\ggt{132}{51}$ ist mit 3 bestimmt. Es soll nun versucht werden, ein $x$ und $y$
  zu finden, sodass $3 = x \cdot 132 + y \cdot 51$. Aus der vorletzten Zeile weiß man:
  \begin{equation*}
    3 = 21 - 2 \cdot 9
  \end{equation*}
  Wie kam die 9 zustande? Aus der Zeile darüber mit $9 = 30 - 21$.
  Eingesetzt und zusammengefasst:
  \begin{equation*}
    3 = 21 - 2 \cdot (30 - 21) = 21 - 2 \cdot 30 + 2 \cdot 21 = -2 \cdot 30 + 3 \cdot 21
  \end{equation*}
  Wie kam die 21 zustande? Erneut aus der Zeile darüber mit $21 = 51 - 30$.
  Eingesetzt, zusammengefasst und für die letzten beiden Zeilen fortgeführt:
  \begin{align*}
    3 & = -2 \cdot 30 + 3 \cdot (51 - 30) =
    -2 \cdot 30 + 3 \cdot 51 - 3 \cdot 30 = 3 \cdot 51 - 5 \cdot 30                          \\
    3 & = 3 \cdot 51 - 5 \cdot (132 - 2 \cdot 51) = 3 \cdot 51 - 5 \cdot 132 + 10 \cdot 51 =
    -5 \cdot 132 + 13 \cdot 51
  \end{align*}
  Man erhält die gewünschte Gleichung.
\end{example}

\noindent
Das Verfahren kann erneut allgemein betrachtet werden. Durch Rückwärtsarbeiten erhält
man verschiedene Gleichungen der folgenden Form:
\begin{align*}
  \ggt{a}{b} & = x_i \cdot a_i + y_i \cdot b_i                 \\
  \ggt{a}{b} & = x_{i+1} \cdot a_{i+1} + y_{i+1} \cdot b_{i+1}
\end{align*}
Außerdem kennen wir die Gleichungen aus \eqref{eq:euklid}:
\begin{align*}
  a_i                      & = q_i \cdot b_i + r_i             \\
  a_{i+1}                  & = q_{i+1} \cdot b_{i+1} + r_{i+1} \\
  \text{mit} \quad a_{i+1} & = b_i                             \\
  b_{i+1}                  & = r_i = a_i - q_i \cdot b_i
\end{align*}
Einsetzen und umformen:
\begin{align*}
  \ggt{a}{b} & = x_i \cdot a_i + y_i \cdot b_i                               \\
             & = x_i \cdot (q_i \cdot b_i + r_i) + y_i \cdot a_{i+1}         \\
             & = x_i \cdot (q_i \cdot a_{i+1} + b_{i+1}) + y_i \cdot a_{i+1} \\
             & = x_i \cdot b_{i+1} + (y_i + x_i \cdot q_i) \cdot a_{i+1}
\end{align*}
Es ergeben sich die folgenden Regeln:
\begin{align}
  x_i & = y_{i+1} \\
  \begin{split}
    y_i + x_i \cdot q_i &= x_{i+1} \\
    y_i &= x_{i+1} - x_i \cdot q_i \\
    y_i &= x_{i+1} - y_{i+1} \cdot q_i
  \end{split}
\end{align}
Dies resultiert im erweiterten Euklidischen Algorithmus, welcher in Tabellenform
einfach und schnell durchführbar ist:

\begin{table}[h]
  \caption{Erweiterter Euklidischer Algorithmus}
  \centering
  \begin{tabular}{|l|l|l|l|l|l|l|l|}
    \hline
    $a$ & $b$ & $q$ & $r$ & $x$                & $y$                                                              & Kontrolle                        \\ \hline
    132 & 51  & 2   & 30  & -5                 & $\hl{Magenta}{3} + \hl{orange}{5} \cdot 2 = 13$                  & $3 = -5 \cdot 132 + 13 \cdot 51$ \\ \hline
    51  & 30  & 1   & 21  & \hl{Magenta}{3}    & $\hl{RoyalBlue}{-2} - \hl{Magenta}{3} \cdot 1 = \hl{orange}{-5}$ & $3 = 3 \cdot 51 - 5 \cdot 30$    \\ \hline
    30  & 21  & 1   & 9   & \hl{RoyalBlue}{-2} & $\hl{Green}{1} + \hl{RoyalBlue}{2} \cdot 1 = \hl{Magenta}{3} $   & $3 = -2 \cdot 30 + 3 \cdot 21$   \\ \hline
    21  & 9   & 2   & 3   & \hl{Green}{1}      & $\hl{Red}{0} - \hl{Green}{1}  \cdot 2 = \hl{RoyalBlue}{-2}$      & $3 = 1 \cdot 21 - 2 \cdot 9$     \\ \hline
    9   & 3   & 3   & 0   & \hl{Red}{0}        & \hl{Green}{1}                                                    & $3 = 0 \cdot 9 + 1 \cdot 3$      \\ \hline
  \end{tabular}
\end{table}

\noindent
Es soll nun gezeigt werden, wie der erweiterte Euklidischen Algorithmus verwendet werden kann,
um multiplikative Inverse zu berechnen. Es soll das Inverse $a \mod{m}$ bestimmt werden, wobei
$m > a$. Das Inverse existiert genau dann, wenn $\ggt{m}{a} = 1$, dies bedeutet es gibt
eine Gleichung der Form $x \cdot m + y \cdot a = 1$. Stellt man diese Gleichung
als Kongruenzrelation Modulo $m$ dar, erhalten wir:
\begin{align*}
  x \cdot m + y \cdot a & = 1               \\
  x \cdot m + y \cdot a & \equiv 1 \pmod{m} \\
  x \cdot 0 + y \cdot a & \equiv 1 \pmod{m} \\
  a \cdot y             & \equiv 1 \pmod{m}
\end{align*}
Die letzte Zeile ist genau die Definition des Inversen \eqref{eq:inverse}, welches
mit $y$ bestimmt wurde.
\begin{example}
  Es soll das Inverse $21^{-1} \pmod{89}$ bestimmt werden. Die Zahlen 21 und 89 sind
  teilerfremd, d.\,h. das Inverse existiert und es gilt
  $\ggt{89}{21} = 1 = x \cdot 89 + y \cdot 21$. Der Euklidische
  Algorithmus kann tabellarisch durchgeführt werden:
  \begin{center}
    \centering
    \begin{tabular}{|l|l|l|l|l|l|l|}
      \hline
      $a$ & $b$ & $q$ & $r$ & $x$ & $y$                  \\ \hline
      89  & 21  & 4   & 5   & -4  & $1 + 4 \cdot 4 = 17$ \\ \hline
      21  & 5   & 4   & 1   & 1   & $0 - 1 \cdot 4 = -4$ \\ \hline
      5   & 1   & 5   & 0   & 0   & 1                    \\ \hline
    \end{tabular}
  \end{center}
  \noindent
  Wir erhalten den größten gemeinsamen Teiler als Linearkombination:
  \begin{equation*}
    -4 \cdot 89 + 17 \cdot 21 = 1
  \end{equation*}
  Es folgt hieraus: Das Inverse von 21 in $\mathbb{Z}_{89}$ beträgt 17. Dieses Ergebnis kann durch nachrechnen
  verifiziert werden:
  \begin{equation*}
    17 \cdot 21 = 357 \equiv 1 \pmod{89} \qedhere
  \end{equation*}
\end{example}

\noindent
Eine rekursive Implementierung des erweiterten Euklidischen Algorithmus ist in
\autoref{alg:ggt-e} zu sehen, das Symbol $\div$ bezeichnet die Ganzzahldivision
zweier Zahlen $a$ und $b$.
Wie zuvor terminiert das Verfahren nachdem das erste
Mal ein Rest von null berechnet wurde.
\begin{singlespace}
  \begin{algorithm}
    \DontPrintSemicolon
    \KwIn{zwei ganze Zahlen $a$ und $b$}
    \KwOut{$\ggt{a}{b}$ als auch $x$ und $y$, sodass $\ggt{a}{b} = x \cdot a + y \cdot b$}
    \BlankLine
    \Fn(){$ggT(a,b)$}{
      $r \leftarrow a \mod{b}$\;
      $q \leftarrow a \div b$\;
      \If(){$r = 0$}{
        \Return{$(0,1,b)$}
      }
      $(x,y,ggt) \leftarrow ggT(b,r)$\;

      \Return{$(y,x - y \cdot q, ggt)$}
    }
    \caption{Erweiterter Euklidischer Algorithmus}
    \label{alg:ggt-e}
  \end{algorithm}
\end{singlespace}
