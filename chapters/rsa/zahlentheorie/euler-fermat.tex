\subsection{Satz von Euler und Fermat}
Es werden im Folgenden zwei Sätze vorgestellt von Euler und Fermat, welche
in der asymmetrischen Verschlüsselung sehr hilfreich sind. Es muss zuvor jedoch noch eine
Eigenschaft der Restklassen festgehalten werden:

\begin{satz}[Eindeutigkeit des $ggT$ von Repräsentanten einer Restklasse und dem Modul $m$]
  \label{satz:restklasse-ggt}
  Für einen Modul $m$ und zwei Repräsentanten $a,b \in \mathbb{N}$ derselben
  Restklasse $\vert a \vert_m = \vert b \vert_m$ gilt:
  \begin{equation*}
    \ggt{a}{m} = \ggt{b}{m}
  \end{equation*}
\end{satz}
\begin{proof}
  Es gilt $b = a + t \cdot m$ mit $t \in \mathbb{Z}$. Es sind
  \begin{align*}
    g_1 & = \ggt{a}{m} = x \cdot \hl{Red}{a} + y \cdot \hl{Red}{m}     \\
    g_2 & = \ggt{b}{m} = p \cdot \hl{Green}{b} + q \cdot \hl{Green}{m}
  \end{align*}
  mit $x,y,p,q \in \mathbb{Z}$. Es gilt außerdem allgemeiner: Wenn $c = x \cdot a + y \cdot m$,
  dann ist $c$ ein Vielfaches von $\ggt{a}{m}$.
  Man schreibt
  \begin{align*}
    g_1 & = xa + ym                                    & g_2 & = pb + qm                                \\
        & = x(b - tm) + ym                             &     & = p(a + tm) + qm                         \\
        & = xb - xtm + ym                              &     & = pa + ptm + qm                          \\
        & = x\,\hl{Green}{b} - (xt + y)\,\hl{Green}{m} &     & = p\,\hl{Red}{a} + (pt + q)\,\hl{Red}{m} \\
        & \Rightarrow g_1 \mid g_2                     &     & \Rightarrow g_2 \mid g_1
  \end{align*}
  und es folgt aus der letzten Zeile: $g_1 = g_2$.
\end{proof}


\begin{satz}[Der Satz von Euler]
  Es seien $a,m \in \mathbb{N}^\times$ mit $a \perp m$, dann gilt:
  \begin{equation*}
    a^{\phif{m}} \equiv 1 \pmod{m}
  \end{equation*}
\end{satz}
\begin{proof}
  Mit gleichem Grundgedanken in \parencite{SITE:phi-euler-fermat} und
  \parencite[187-188]{BOOK:numberTheory}. \\
  Es ist $n := \phif{m}$. Es seien $x_1,x_2,\ldots,x_n$ die $n$ verschiedenen zu
  $m$ teilerfremden Zahlen aus der Menge $\{1,2,\ldots,m\}$.
  Mit einer vorgegeben Zahl
  $a \in \mathbb{N}^\times \wedge a \perp m$ bilden wir die Produkte $ax_1,ax_2,\ldots,ax_n$.
  Wir betrachten ein Beispiel um eine Vermutung zu bestätigen:
  \begin{example}
    Es ist $\phif{8} = |\{1,3,5,7\}| = 4$. Wir wählen $a = 3$ und bilden in $\mathbb{Z}_8$ die Produkte:
    $3\cdot \{1,3,5,7\} = \{3,9,15,21\} \Rightarrow \{3,1,7,5\}$. Die Multiplikation von $a$ erzeugt eine
    Permutation der Menge $\{1,3,5,7\}$.
  \end{example}
  \noindent
  Es sind $|x_1|_m,\ldots,|x_n|_m$ genau alle zu $m$ teilerfremden Restklassen.
  \footnote{
    Nach \autoref{satz:restklasse-ggt} wissen wir, ist eine Zahl einer Restklasse teilerfremd zu $m$,
    gilt dies auch für alle anderen Zahlen der Restklasse.
  }
  Es gilt $x_i \perp m$ und $a \perp m$, weshalb auch das Produkt $ax_i$ teilerfremd zu $m$ sein muss.\\
  Es folgt $ax_i \mod{m} = x_j$ und wir halten fest:
  \begin{equation}
    \label{proof:satz-euler-1}
    \forall ax_i \, \exists x_j: ax_i \equiv x_j \pmod{m}
  \end{equation}
  Es bleibt zu zeigen, dass in \eqref{proof:satz-euler-1} keine Doppelungen entstehen,
  also alle $ax_i$ Teil unterschiedlicher Restklassen sind. Es muss gelten:
  \begin{equation*}
    ax_i \not\equiv ax_j \pmod{m} \qquad i \neq j
  \end{equation*}
  Angenommen $ax_i \equiv ax_j \pmod{m}$ für $i \neq j$, dann gilt nach der Kürzungsregel
  \footnote{
    Da das Inverse wegen $a \perp m$ existiert, kann mit diesem multipliziert werden und die $a$'s
    fallen weg.
  }
  $x_i \equiv x_j \pmod{m}$, was nicht sein kann.
  Wegen $x_1,x_2,\ldots,x_n \perp m$ liefert die Kürzungsregel, wenn man noch
  $n = \phif{m}$ beachtet, die ursprüngliche Behauptung:
  \begin{align*}
    ax_1 \cdot ax_2 \cdot \ldots \cdot ax_n & \equiv
    x_1 \cdot x_2 \cdot \ldots \cdot x_n  \pmod{m}                       \\
    a^{\phif{m}}                            & \equiv 1 \pmod{m} \qedhere
  \end{align*}
\end{proof}

\noindent
Der Kleine Satz von Fermat kann jetzt als Spezialform des Satzes von Euler (mit $m = p$) direkt
aufgeschrieben werden:

\begin{satz}[Kleiner Satz von Fermat]
  Es seien $a \in \mathbb{N}^\times$ und $p \in \mathbb{P}$ mit $a \perp p$, dann gilt:
  \begin{equation*}
    a^{p-1} \equiv 1 \pmod{p}
  \end{equation*}
\end{satz}

\noindent
Der Kleine Satz von Fermat wird in der Kryptografie unter anderem dafür verwendet,
Primzahltests durchzuführen.