\subsection{Die Eulersche Phi-Funktion}
Die Eulersche Phi-Funktion befasst sich mit dem auf den ersten Blick seltsamen
Problem, die Anzahl an teilerfremden Zahlen in einer Menge zu finden. Die Regeln
und Sätze welche von der zahlentheoretischen Funktion abgeleitet werden können, sind jedoch sehr
hilfreich in der asymmetrischen Verschlüsselung und insbesondere für das
RSA-Verfahren. Die Eulersche Phi-Funktion ist folgendermaßen definiert
\parencite[165]{BOOK:crypto} \parencite{SITE:phi-euler-fermat}:

\begin{definition}[Eulersche Phi-Funktion]
  Die Anzahl der zu $m \in \mathbb{N}^\times$ teilerfremden Zahlen aus $\{1,2,\dots,m\}$,
  ist gekennzeichnet als $\phif{m}$ und heißt Eulersche Phi-Funktion. $\phif{m}$
  kann formal beschrieben werden als die Mächtigkeit der Menge:
  \begin{equation*}
    \phif{m} = \vert \{x \in \mathbb{N} \mid 1 \leq x \leq m \wedge \ggt{m}{x} = 1 \} \vert
  \end{equation*}
\end{definition}

\begin{example}
  \begin{align*}
    \phif{5}  & = \vert \{1,2,3,4\} \vert = 4                      \\
    \phif{6}  & = \vert \{1,5\} \vert = 2                          \\
    \phif{20} & = \vert \{1,3,7,9,11,13,17,19\} \vert = 8 \qedhere
  \end{align*}
\end{example}

\noindent
Die naive Berechnung der Eulerschen Phi-Funktion, alle Zahlen zu durchlaufen und den
größten gemeinsamen Teiler zu bestimmen, ist sehr langsam und für große Zahlen, wie sie
in der asymmetrischen Verschlüsselung verwendet werden, nicht möglich. Es existiert jedoch
eine Beziehung, mit welcher $\phif{m}$ bestimmt werden kann, unter der
Bedingung, dass die Primfaktorzerlegung von $m$ bekannt ist. Diese lässt sich aus
den folgenden Sätzen ableiten, welche hier im einzelnen nicht noch einmal bewiesen werden
sollen \parencite{SITE:phi-euler-fermat}:
\begin{enumerate}[ref=(\arabic*)]
  \item Es sei $p \in \mathbb{P}$, dann gilt: $\phif{p} = p - 1$. \label{enum:phi1}
  \item Es sei $p \in \mathbb{P}$, dann gilt: $\phif{p^n} = p^n - p^{n-1}$. \label{enum:phi2}
  \item Es sei $\ggt{n}{m} = 1$, dann gilt:
        $\phif{n \cdot m} = \phif{n} \cdot \phif{m}$. \label{enum:phi3}
\end{enumerate}

\begin{satz}[Formel zur Eulerschen Phi-Funktion]
  Es sei
  \begin{equation*}
    n = p_1^{m_1} \cdot p_2^{m_2} \cdot \ldots \cdot p_r^{m_r} = \prod_{i=1}^{r} p_i^{m_i}
  \end{equation*}
  die Primfaktorzerlegung einer natürlichen Zahl mit $p_i$ unterschiedlichen Primzahlen
  und Exponenten $m_1 \geq 1,\ldots, m_r \geq 1$. Dann ist:
  \begin{equation*}
    \phif{n} = \prod_{i=1}^{r} p_i^{m_i} - p_i^{m_i - 1} =
    n \cdot \prod_{i=1}^{r} 1 - \frac{1}{p_i}
  \end{equation*}
\end{satz}

\begin{proof}
  Da die Primfaktoren einer Zahl teilerfremd sind, kann nach Regel \ref{enum:phi3}
  und \ref{enum:phi2} geschrieben werden:
  \begin{align*}
    \phif{n} \numeq{\ref{enum:phi3}} \prod_{i=1}^{r} \phif{p_i^{m_i}}
    \numeq{\ref{enum:phi2}} \prod_{i=1}^{r} p_i^{m_i} - p_i^{m_i - 1}
     & = \prod_{i=1}^{r} p_i^{m_i} \cdot (1 - \frac{1}{p_i})               \\
     & = \prod_{i=1}^{r} p_i^{m_i} \cdot \prod_{i=1}^{r} 1 - \frac{1}{p_i} \\
     & = n \cdot \prod_{i=1}^{r} 1 - \frac{1}{p_i} \qedhere
  \end{align*}
\end{proof}

\begin{example}
  \begin{align*}
    \phif{6 = 2 \cdot 3}        & = 6 \cdot (1 - \frac{1}{2}) \cdot (1 - \frac{1}{3}) =
    6 \cdot \frac{2}{6} = 2                                                              \\
    \phif{20 = 2^2 \cdot 5}     & = 20 \cdot (1 - \frac{1}{2}) \cdot (1 - \frac{1}{5}) =
    20 \cdot \frac{4}{10} = 8                                                            \\
    \phif{1000 = 2^3 \cdot 5^3} & = 1000 \cdot \frac{4}{10} = 400 \qedhere
  \end{align*}
\end{example}