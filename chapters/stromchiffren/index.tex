\chapter{Stromchiffren und Blockchiffren}

\begin{figure}[h]
  \centering
  \begin{tikzpicture}
    [every node/.style={draw, inner xsep=4mm, inner ysep=2mm, rounded corners=2mm, align=center}]
    \node(root) {Kryptografie};

    \node[below left=of root] (A) {Symmetrische \\ Chiffren};
    \node[below=of root] (B) {Asymmetrische \\ Chiffren};
    \node[below right=of root] (C) {Protokolle};

    \node[below left=1cm and -1cm of A] (D) {Stromchiffren};
    \node[below right=1cm and -1cm of A] (E) {Blockchiffren};

    \coordinate[below=0.5cm of root] (x);
    \draw (root) -- (x);
    \draw [->] (x) -| (A);
    \draw [->] (x) -| (B);
    \draw [->] (x) -| (C);

    \coordinate[below=0.5cm of A] (y);
    \draw (A) -- (y);
    \draw [->] (y) -| (D);
    \draw [->] (y) -| (E);
  \end{tikzpicture}
  \caption{Unterteilung der Symmetrischen Chiffren \parencite[29]{BOOK:crypto}}
  \label{fig:sym-ciphers-overview}
\end{figure}

\noindent
Werfen wir in \autoref{fig:sym-ciphers-overview} einen genaueren Blick auf die Algorithmen
der Kryptografie, stellen wir fest: Das Gebiet
der symmetrischen Verschlüsselungsverfahren kann unterteilt werden in Stromchiffren
und Blockchiffren.
In diesem Kapitel sollen anhand von Stromchiffren einige Definitionen eingeführt
und der Unterschied zu den Blockchiffren erläutert werden.
Außerdem wird gezeigt, welche Rolle hierbei die Zufallszahlengeneratoren spielen.

\section{Ein Vergleich der Verfahren}
Die symmetrischen Verschlüsselungsverfahren sind unterteilt in Strom- und Blockchiffren.
Während beide Verfahren das gleiche Ziel verfolgen, Informationen zu verschlüsseln,
ist die jeweilige
Methode eine unterschiedliche. Stromchiffren verschlüsseln jedes Bit im Klartext einzeln,
während Blockchiffren pro Durchlauf mehrere Bits verschlüsseln können.
\autoref{fig:stream-vs-block-cipher} zeigt diesen prinzipiellen Unterschied, für den Fall, dass
$n$ Bits verschlüsselt werden sollen.

\begin{figure}[h]
  \centering
  \begin{tikzpicture}
    [box/.style={draw, inner xsep=4mm, inner ysep=2mm}]

    \node[box] (stream) {Stromchiffre};
    \node[font=\small, above=of stream] (k1) {$k$};
    \node[below=1pt of stream] {(a)};

    \node[box, right=5cm of stream] (block) {Blockchiffre};
    \node[font=\small, above=of block] (k2) {$k$};
    \node[below=1pt of block] {(b)};

    \draw[->] (k1) to (stream);
    \draw[<-] (stream.west) to node[pos=1, above, font=\small] {$x_1,x_2,\dots,x_n$} ++(-1.5cm,0);
    \draw[->] (stream.east) to node[pos=1, above, font=\small] {$y_1,y_2,\dots,y_n$} ++(1.5cm,0);

    \draw[->] (k2) to (block);
    \draw[<-] (block.west) to
    node[pos=1, above, font=\small, align=center] {
      $x_1$ \\
      $x_2$ \\
      \vdots \\
      $x_n$
    } ++(-1.5cm,0);
    \draw[->] (block.east) to
    node[pos=1, above, font=\small, align=center] {
      $y_1$ \\
      $y_2$ \\
      \vdots \\
      $y_n$
    } ++(1.5cm,0);
  \end{tikzpicture}
  \caption{Das Prinzip der Verschlüsselung von $n$ Bits mittels Strom- (a) und
    Blockchiffre (b) \parencite[30]{BOOK:crypto}}
  \label{fig:stream-vs-block-cipher}
\end{figure}

\noindent
Blockchiffren arbeiten mit Datenblöcken fester Länge, wobei ein Großteil der
relevanten Algorithmen eine Eingangsweite von 64 oder 128 Bit besitzen.
Prominente Beispiele sind Verfahren wie der \acl{des}
(\acs{des}, Blocklänge 64 Bit, \cite[55-58]{BOOK:crypto}) oder der \acl{aes}
(\acs{aes}, Blocklänge 128 Bit, \cite[87-90]{BOOK:crypto}).
Eine grundlegende Idee für das Entwerfen starker Blockchiffren, ist das Prinzip
der Konfusion (engl. \textit{confusion}) und
Diffusion (engl. \textit{diffusion}) \parencite[57]{BOOK:crypto}:
\begin{enumerate}
  \item Die \textbf{Konfusion} hat die Aufgabe, den
        Zusammenhang von Schlüssel und Geheimtext unklar zu machen. Ein
        beliebtes Vorgehen ist die Bitsubstitution, welche sowohl in \acs{des} als auch
        \acs{aes} verwendet wird.
  \item Die \textbf{Diffusion} hat die Aufgabe, den Einfluss
        an einer Stelle des Klartextes, über den gesamten Geheimtext zu verteilen. Es wird
        das Ziel verfolgt, statistische Eigenschaften zu verstecken.
        Das Ändern von einer Stelle im Klartext
        soll im Durchschnitt eine Änderung des halben Geheimtextes bewirken. Eine einfache
        Operation für Diffusion ist die Bitpermutation, welche häufig in \acs{des} eingesetzt wird.
        \acs{aes} verwendet eine komplexere MixColumn-Transformation.
\end{enumerate}
Chiffren, welche nur eine der beiden Operationen anwenden, wie die alleinige Konfusion
mit der Cäsar-Chiffre, sind nicht sicher. Die heutigen Blockchiffren
arbeiten iterativ in aufeinanderfolgenden Runden, die gleich aufgebaut sind. In jeder Runde wird eine
Rundenfunktion angewandt, welche sowohl Konfusion als auch Diffusion durchführt.
Generell lassen sich die folgenden drei Punkte zusammenfassen \parencite[31]{BOOK:crypto}:

\begin{enumerate}
  \item Blockchiffren werden in der Praxis vor allem für die Verschlüsselung im Internet
        häufiger eingesetzt als Stromchiffren.
  \item Stromchiffren sind oft klein und schnell und deshalb attraktiv für Anwendungen,
        bei denen vergleichsweise wenig Rechenleistung zur Verfügung steht, beispielsweise bei
        Mobilgeräten oder anderen eingebetteten Systemen. Ein bekanntes Verfahren ist der
        A5/1 Algorithmus, welcher Teil des GSM-Mobilfunkstandards ist und die Gesprächsdaten
        an der Luftschnittstelle verschlüsselt. A5/1 gilt aufgrund der kurzen Schlüssellänge von
        64 Bit nicht mehr als uneingeschränkt sicher, erschwert aber dennoch das einfache Abhören.
  \item In der Vergangenheit galt der generelle Gedanke, dass Stromchiffren
        effizienter seien als Blockchiffren.
        Im Allgemeinen gilt diese Annahme heutzutage jedoch nicht mehr. Moderne Verfahren wie AES
        können sowohl in Hardware als auch Software sehr effizient implementiert werden.
\end{enumerate}

\section{Die Ver- und Entschlüsselung mit Stromchiffren}
Stromchiffren verschlüsseln jedes Bit im Klartext einzeln. Hierfür
wird durch einen Schlüs\-sel ein geheimer Bitstrom errechnet, welcher paarweise mit den Bits
des Klartextes kombiniert wird. Die Ver- und Entschlüsselung ist verblüffend einfach, es handelt
sich in beide Richtungen um eine einfache Addition im Ring $\mathbb{Z}_2$.

\begin{definition}[{Ver- und Entschlüsselung mit Stromchiffren, \cite[31]{BOOK:crypto}}]
  Es seien $x_i,y_i,k_i \in \{0,1\}$ die einzelnen Bit aus Klartext, Geheimtext und Schlüs\-selstrom.
  Es gilt:
  \begin{description}
    \item[Verschlüsselung:] $y_i = e_{k_i}(x_i) \equiv x_i + k_i \pmod{2}$
    \item[Entschlüsselung:] $x_i = d_{k_i}(y_i) \equiv y_i + k_i \pmod{2}$
  \end{description}
\end{definition}

\noindent
Betrachteten wir die Ver- und Entschlüsselungsfunktion, fallen drei Aspekte auf, welche
besprochen werden müssen \parencite[31-34]{BOOK:crypto}:

\begin{enumerate}
  \item Warum sind Verschlüsselung und Entschlüsselung die selbe Funktion?
  \item Warum ist einfache Addition im Ring $\mathbb{Z}_2$ eine gute Verschlüsselung?
  \item Was sind die Eigenschaften der Schlüsselstrombit $k_i$?
\end{enumerate}

\paragraph{Warum sind Verschlüsselung und Entschlüsselung die selbe Funktion?}\mbox{}\\
Bis auf spezielle Kürzungsregeln erlauben Kongruenzen (ganz analog wie mit Gleichungen)
das Ausführen der elementaren Rechenoperationen \parencite[181-183]{BOOK:numberTheory}.
Mit diesem Wissen kann die Verschlüsselungsfunktion durch einfaches Umformen in die
Entschlüsselungsfunktion überführt werden:
\begin{proof}
  \begin{align*}
    y_i  & \equiv x_i + s_i \pmod{2}          \\
    -x_i & \equiv -y_i + s_i \pmod{2}         \\
    x_i  & \equiv y_i + s_i \pmod{2} \qedhere
  \end{align*}
\end{proof}
\noindent
Im letzten Schritt wird erneut vom Wechsel innerhalb der Restklasse Gebrauch gemacht.
Es gilt: $-1 \equiv 1 \pmod{2}$.

\paragraph{Warum ist einfache Addition im Ring $\mathbb{Z}_2$ eine gute Verschlüsselung?}\mbox{}\\
Das Rechnen in $\mathbb{Z}_2$ liefert aufgrund der Division mit Rest nur Ergebnisse in
der Menge $\{0,1\}$. Dieses Verhalten ist sehr hilfreich, denn es ermöglicht das Ausdrücken der
Rechenregeln durch einfache boolesche Algebra.
Betrachten wir die Wahrheitstabelle (\ref{tab:truth-table-addition-mod-2})
der Addition in $\mathbb{Z}_2$, kann sofort eine weitere Beobachtung gemacht werden:
Die Addition Modulo 2 ist äquivalent zu der Exklusiv-Oder-Verknüpfung durch ein
XOR-Gatter:

\begin{table}[h]
  \centering
  \caption{Wahrheitstabelle der Addition Modulo 2}
  \begin{tabular}{cc|c}
    $x_i$ & $k_i$ & $y_i \equiv x_i + k_i \pmod{2}$ \\ \hline
    0     & 0     & 0                               \\
    0     & 1     & 1                               \\
    1     & 0     & 1                               \\
    1     & 1     & 0                               \\
  \end{tabular}
  \label{tab:truth-table-addition-mod-2}
\end{table}

\noindent
Das XOR-Gatter spielt eine wesentliche Rolle in vielen kryptografischen Verfahren.
Es besitzt besondere Eigenschaften, welche es von anderen Logikgattern
unterscheidet und diese sollen jetzt untersucht werden.
Angenommen es soll das Klartextbit $x_i = 0$ verschlüsselt werden. In der Wahrheitstabelle
(\ref{tab:truth-table-xor}) befindet
man sich demnach in der ersten oder zweiten Zeile.
Je nach Schlüsselbit $k_i$ ist das Geheimtextbit $y_i$ gegeben als
$y_i = 0 \oplus 0 \vee 0 \oplus 1 = 0 \vee 1$.
Verhalten sich die Schlüsselbit unvorhersehbar, d. h. sie sind mit genau 50-prozentiger
Wahrscheinlichkeit null oder eins, ist es nur durch das Chiffrat nicht möglich, auf
den Klartext zu schließen. Zu jedem Zeitpunkt hat ein Angreifer nur eine
50-prozentige Chance, den richtigen Klartext zu erraten.
Gleichermaßen kann argumentiert werden, sei $x_i = 1$.
Diese Symmetrie unterscheidet das XOR-Gatter von anderen Logikgattern, zusätzlich ist es die einzige
Verknüpfung, welche durch doppeltes Anwenden invertierbar ist.
\autoref{fig:enc-dec-stream-cipher} zeigt den Ver- und Entschlüsselungsprozess
bei Stromchiffren.

\begin{table}
  \centering
  \caption{Wahrheitstabelle der Exklusiv-Oder-Verknüpfung}
  \begin{tabular}{cc|c}
    $x_i$      & $k_i$      & $y_i = x_i \oplus k_i$ \\ \hline
    \textbf{0} & \textbf{0} & \textbf{0}             \\
    \textbf{0} & \textbf{1} & \textbf{1}             \\
    1          & 0          & 1                      \\
    1          & 1          & 0                      \\
  \end{tabular}
  \label{tab:truth-table-xor}
\end{table}

\begin{figure}[h]
  \centering
  \begin{tikzpicture}
    \node[XOR, scale=2] (xor-e) {};
    \node[above=of xor-e] (s1) {$k_i$};

    \node[unsecure channel, right=1.5cm of xor-e]
    (K) {\small unsicherer Kanal \\ \small (z.B. Internet)};

    \node[XOR, scale=2, right=1.5cm of K] (xor-d) {};
    \node[above=of xor-d] (s2) {$k_i$};

    \draw[<-] (xor-e.west) to node[pos=1, above] {$x_i$} ++(-1.5cm,0);
    \draw[->] (s1) to (xor-e);
    \draw[->] (xor-e) to node[above] {$y_i$} (K);

    \draw[->] (K) to node[above] {$y_i$} (xor-d);
    \draw[->] (s2) to (xor-d);
    \draw[->] (xor-d) to node[pos=1, above] {$x_i$} ++(1.5cm,0);
  \end{tikzpicture}
  \caption{Der Ver- und Entschlüsselungsprozess bei Stromchiffren}
  \label{fig:enc-dec-stream-cipher}
\end{figure}

\noindent
In einem Beispiel soll jetzt ein einfacher Informationsaustausch demonstriert werden:

\newcommand{\streamencryption}{
  \begin{aligned}
    x_0,\dots,x_7 = 01010 & 000_2 = 80_{10} = \text{\texttt{P}}  \\
    \oplus                &                                      \\
    k_0,\dots,k_7 = 00111 & 010_2                                \\
    y_0,\dots,y_7 = 01101 & 010_2 = 106_{10} = \text{\texttt{j}}
  \end{aligned}
}

\newcommand{\streamdecryption}{
  \begin{aligned}
    y_0,\dots,y_7 = 01101 & 010_2 = 106_{10} = \text{\texttt{j}} \\
    \oplus                &                                      \\
    k_0,\dots,k_7 = 00111 & 010_2                                \\
    x_0,\dots,x_7 = 01010 & 000_2 = 80_{10} = \text{\texttt{P}}
  \end{aligned}
}

\newcommand{\streamarrow}{\tikz{\draw[->] (0,0) to
    node[above] {$\text{\texttt{j}} = 01101010_2$} (2,0);}}

\begin{example}
  Alice möchte Bob eine Nachricht senden. Sie verschickt den Buchstaben \texttt{P},
  wobei dieser zur Verschlüsselung in
  ASCII-Zeichenkodierung angegeben ist $\text{\texttt{P}} = 80_{10} = 01010000_2$. Der
  Schlüsselstrom wird außerdem angegeben, es seien
  $(k_0,k_1,\dots,k_7) \allowbreak = 00111010_2$.
  \newpage
  \begin{table*}
    \centering
    \begin{tabular}{lc}
      $\streamencryption$              & \textbf{Alice} \\
                                       &                \\
      \multicolumn{1}{c}{\streamarrow} & \textbf{Oscar} \\
                                       &                \\
      $\streamdecryption$              & \textbf{Bob}
    \end{tabular}
  \end{table*}

  \noindent
  Vor der Übertragung wird der Großbuchstabe \texttt{P} durch die XOR-Verknüpfung
  in den Kleinbuchstaben \texttt{j}
  umgewandelt. Der Gegenspieler Oscar kann mit dieser Information aufgrund der oben beschriebenen
  Eigenschaften nur wenig anfangen.
\end{example}

\noindent
Stromchiffren erscheinen fast zu gut, um wahr zu sein, doch wie später gezeigt wird,
gibt es aus auch hier Einschränkungen, welche in der Praxis entstehen.
Es bleibt die Letzte der drei Fragestellungen zu beantworten.

\paragraph{Was sind die Eigenschaften der Schlüsselstrombit?}\mbox{}\\
Die Sicherheit von Stromchiffren hängt vollständig von der Qualität des Schlüsselstroms ab.
Das Generieren dieser Folge ($k_1,k_2,\dots,k_n$) bildet daher
die zentrale Fragestellung der Verschlüsselungsmethode.
Es wurde gezeigt, dass der Schlüsselstrom wie eine zufällige
Folge von Bit aussehen muss. Im nächsten Abschnitt soll deshalb das Thema der
Zufallszahlen diskutiert werden.

\section{Zufallszahlengeneratoren}
Die Unvorhersehbarkeit des Schlüsselstroms ist die wesentliche
Eigenschaft von Stromchiffren. Im Folgenden sollen einige
Formen von Zufallszahlengeneratoren (engl. \acp{rng}) vorgestellt
und deren Unterschiede beschrieben werden
\parencite[35-36]{BOOK:crypto} \parencite{SITE:randomorg}.

\paragraph{Echte Zufallszahlengeneratoren}
Echte Zufallszahlengeneratoren (engl. \acp{trng})
erzeugen Zahlen, welche in keiner Weise vorhersehbar und reproduzierbar sind. Notiert man
beispielsweise das Ergebnis von 100 Münzwürfen und erzeugt eine Folge von 100 Bit, ist es quasi
unmöglich, dieselbe Sequenz ein zweites Mal zu erzeugen. Die Wahrscheinlichkeit,
dass dies passieren würde, beträgt 1 in $2^{100}$, was verschwindend gering ist.
Echte Zufallszahlen basieren auf physikalischen Prozessen wie das Werfen einer Münze, Würfeln,
radioaktiver Zerfall oder atmosphärisches Rauschen. \acp{trng} werden in der Kryptografie und
Schlüsselerzeugung häufig eingesetzt.

\paragraph{Pseudozufallszahlengeneratoren}
Pseudozufallszahlengeneratoren (engl. \acp{prng})
generieren Zahlenfolgen basierend auf einem Startwert, welcher im Englischen oft als
\enquote{\textit{seed}} bezeichnet wird. Die Zahlen eines \ac{prng} sind nicht
in der Art zufällig, wie man es erwarten könnte.
Obwohl sie aussehen wie echte Zufallszahlen, werden sie durch
mathematische Formeln errechnet und sind deterministisch, d.\,h. dieselbe Sequenz
kann zu einem späteren Zeitpunkt reproduziert werden.
In der Kryptografie können \acp{prng} aufgrund dieser Eigenschaft
nicht ohne Weiteres eingesetzt werden.
Dennoch haben sie außerhalb der Kryptografie weitreichende Anwendungsgebiete,
beispielsweise in der Simulation oder während des Testens von Software und Hardware.
Determinismus ist in diesen Bereichen oftmals eine gewünschte Eigenschaft.

\paragraph{Kryptografisch sichere Pseudozufallszahlengeneratoren}
Kryptografisch sichere Pseudozufallszahlengeneratoren
(engl. \acp{csprng}) sind \acp{prng} mit
einer zusätzlichen Eigenschaft: \acp{csprng} sind unvorhersehbar. Grob gesprochen
bedeutet dies, dass es rechentechnisch nicht möglich ist, aus $n$ gegeben Schlüsselstrombit
$k_1,k_2,\dots,k_n$, die folgenden Bit $k_{n+1},k_{n+2},\dots,k_{n+m}$ zu berechnen. Außerdem
soll es zeitlich unmöglich sein, eines der vorherigen Bit
$k_{0},k_{-1},\allowbreak\dots,k_{-m}$ zu berechnen.

\section{Das One-Time-Pad}
Es sind nun mit den bisher eingeführten Ideen alle Teile vorhanden, um ein beweisbar
sicheres Verschlüsselungsverfahren zu entwerfen. Ein System heißt beweisbar sicher,
wenn es trotz unendlich vorhandener Rechenleistung nachweislich nicht gebrochen werden kann.
Es kann folgende Definition gegeben werden \parencite[36]{BOOK:crypto}:

\begin{definition}[\textit{Unconditional Security}]
  \enquote{\textit{A cryptosystem is unconditionally or in\-formation-theoretically
      secure if it cannot be broken even with infinite computational resources.}}
\end{definition}

\noindent
Das \ac{otp} ist ein solches Verschlüsselungsverfahren, welches dieses Kriterium
erfüllt. Es ist folgendermaßen definiert \parencite[37]{BOOK:crypto}:

\begin{definition}[One-Time-Pad]
  Eine Stromchiffre heißt One-Time-Pad, wenn die folgenden Kriterien eingehalten werden:
  \begin{enumerate}
    \item Der Schüsselstrom $k_1,k_2,\dots,k_n$ wird durch einen \ac{trng} generiert.
    \item Nur vertrauenswürdige Gesprächspartner kennen den Schüsselstrom.
    \item Jedes Schlüsselstrombit $k_i$ wird nur einmal verwendet.
  \end{enumerate}
  Das One-Time-Pad ist beweisbar sicher.
\end{definition}
\noindent
Zu jedem Bit im Geheimtext gibt es eine Gleichung der Form:
\begin{align*}
  y_0   \equiv x_0 & + k_0 \pmod{2} \\
  y_1   \equiv x_1 & + k_1 \pmod{2} \\
  \vdots           &                \\
  y_n   \equiv x_n & + k_n \pmod{2} \\
\end{align*}
\noindent
Stammt der Schlüsselstrom von einem \ac{trng} können diese Gleichungen nicht gelöst werden.
Für alle $y_i$ gibt es genau zwei Lösungen mit gleich großer Wahrscheinlichkeit:
\begin{align*}
  y_i = 0 & \equiv 0_{x_i} + 0_{k_i} \equiv 1_{x_i} + 1_{k_i} \pmod{2} \\
  y_i = 1 & \equiv 0_{x_i} + 1_{k_i} \equiv 1_{x_i} + 0_{k_i} \pmod{2}
\end{align*}

\section{Die praktischen Stromchiffren}
In der Praxis hat das Verschlüsseln mit \ac{otp} natürlich einen Hacken, denn warum sonst
sollten heutzutage andere Verfahren eingesetzt werden?
Der erste Nachteil ist die Anforderung eines \ac{trng}, dieser benötigt spezielle
Hardware und ist in der Regel nicht Teil eines Standardcomputers.
Der zweite und wahrscheinlich größere Nachteil ist die Handhabung des Schlüsselstroms:
Jedes Schlüsselstrombit wird nur einmal verwendet.
Das One-Time-Pad benötigt somit einen Schlüssel, welcher genauso lang ist
wie die Nachricht selbst. Es ist klar, warum diese Anforderung
selbst bei mittelgroßen Nachrichten problematisch wird, zusätzlich muss nach jeder
Übertragung ein neuer Schlüssel ausgetauscht werden. Die Idee von \ac{otp} ist gut,
jedoch müssen für den praktischen Gebrauch einige Modifikationen vorgenommen werden.
Es wird versucht, den Schüsselstrom, welcher aus einer echt zufälligen Quelle stammt,
durch die Ausgabe eines \ac{csprng} zu ersetzen, wobei der Schlüssel $k$
den Startwert des \ac{prng} bildet. Das Prinzip dieser Idee ist in
\autoref{fig:stream-key-gen} zu sehen.

\begin{figure}[h]
  \centering
  \begin{tikzpicture}
    \node[unsecure channel, font=\small] (channel) {unsicherer Kanal \\ (z.B. Internet)};

    \node[above=of channel] (oscar) {Oscar};

    \node[align=center, above=2cm of oscar, inner xsep=-0.7cm, yshift=-0.5cm] (init-key)
    {Initialer Schlüssel \\ (kurz)};

    \node[XOR, scale=2, left=1.5cm of channel] (xor-e) {};
    \node[XOR, scale=2, right=1.5cm of channel] (xor-d) {};

    \node[draw, align=center, above=of xor-e] (key-gen-l) {Schlüsselstrom-\\generator};
    \node[draw, align=center, above=of xor-d] (key-gen-r) {Schlüsselstrom-\\generator};

    \node[above=of key-gen-l, inner sep=5pt] (k-l) {$k$};
    \node[above=of key-gen-r, inner sep=5pt] (k-r) {$k$};

    \coordinate[left=2.5cm of xor-e] (p-left);
    \coordinate[right=2.5cm of xor-d] (p-right);

    \node[above=2cm of p-left] (alice) {Alice};
    \node[above=2cm of p-right] (bob) {Bob};

    \draw [->] (k-l) to (key-gen-l);
    \draw [<-] (xor-e) to node[right] {$s_i$} (key-gen-l);
    \draw [<-] (xor-e.west) to node[pos=1, above] {$x_0,x_1,\dots,x_n$} (p-left);
    \draw [->] (xor-e) to node[above] {$y_i$} (channel);

    \draw [<->] (oscar) to (channel);

    \draw [->] (k-r) to (key-gen-r);
    \draw [<-] (xor-d) to node[right] {$s_i$} (key-gen-r);
    \draw [->] (xor-d.east) to node[pos=1, above] {$x_0,x_1,\dots,x_n$} (p-right);
    \draw [<-] (xor-d) to node[above] {$y_i$} (channel);

    \draw [->, shorten >=0.6cm] (alice) to (p-left);
    \draw [->, shorten >=0.6cm] (bob) to (p-right);

    \draw (init-key)
    edge [->, dashed] (k-l)
    edge [->, dashed] (k-r);
  \end{tikzpicture}
  \caption{Praktische Stromchiffre mit Schlüsselstromerzeugung \parencite[38]{BOOK:crypto}}
  \label{fig:stream-key-gen}
\end{figure}

\noindent
Zunächst ist wichtig festzuhalten, dass
Stromchiffren nicht beweisbar sicher sind. Es ist allerdings so, dass alle in der
Praxis verwendeten Algorithmen (Stromchiffren, Blockchiffren, asymmetrische
Verfahren), keine informationstheoretische Sicherheit aufweisen \parencite[38]{BOOK:crypto}.
Um den Sicherheitsbegriff realistischer zu gestalten, wird der Angreifer in seiner
Rechenleistung beschränkt, man hofft also, dass ein Verfahren berechenbarkeitstheoretisch
sicher ist. Diese Eigenschaft ist von \citeauthor{BOOK:crypto}
folgendermaßen definiert \parencite*[35]{BOOK:crypto}:

\begin{definition}[\textit{Computational Security}]
  \enquote{\textit{A crpytosystem is computationally secure if the best
      known algorithm for breaking it requires at least $t$ operations.}}
\end{definition}

\noindent
Die Definition besagt, dass ein System berechenbarkeitstheoretisch sicher ist, wenn es
keinen bekannten Algorithmus gibt, welcher das System effizient brechen kann.
In anderen Worten ausgedrückt: Es ist nicht effizient möglich, den
Schlüssel anhand des Geheimtextes zu berechnen. Allgemein kann keine
Aussage darüber getroffen werden, ob ein Problem mit bekannten Verfahren
optimal gelöst wird, weshalb die Sicherheit eines Systems nie garantiert,
sondern nur angenommen werden kann. Ein bekanntes Beispiel ist dass des RSA-Kryptosystems
(asymmetrisches Verschlüsselungsverfahren), wessen Schlüssel
berechnet werden kann, durch das Faktorisieren zweier großer
natürlicher Zahlen. Obwohl viele Faktorisierungsverfahren bekannt sind, ist es nicht klar,
ob es andere Verfahren gibt, welche das Problem effizienter lösen können.
Im Fall von symmetrischen Verfahren nimmt man an, dass es keinen besseren Algorithmus
gibt als die vollständige Schlüsselsuche.
