\chapter{Einleitung}
Redet man heutzutage über das Thema Kryptographie, sind im Gespräch wahrscheinlich
Themen wie E-Mail-Verschlüsselung, Internetprotokolle oder Anwendungen im
Bankenwesen. Auch bekannt sind die Angriffe auf kryptographische Systeme,
wie zum Beispiel die Entzifferung der durch die Enigma-Chiffriermaschine
verschlüsselten deutschen Funksprüche während des Zweiten Weltkrieges. Es scheint als
wäre Kryptographie stark mit den modernen elektronischen Kommunikationstechniken
verbunden. Dies ist allerdings nicht so: Frühe Formen der Kryptographie
gehen zurück bis etwa 2000 v. Chr., als bereits im antiken Ägypten neben den
Standard-Hieroglyphen zusätzlich auch \enquote{geheime} Zeichen
verwendet wurden \parencite[2]{BOOK:crypto}. Es werden prinzipiell zwei
unterschiedliche kryptographische Verfahren unterschieden, diese sind
Symmetrische- und Asymmetrische Algorithmen. Die symmetrische Verschlüsselung ist
seit langer Zeit ein fester Bestandteil der Kryptographie, mit bekannten historischen
Verfahren wie die Cäsar-Chiffre welche bereits im antiken Rom für das Verschlüsseln von
Nachrichten verwendet wurde. Asymmetrische Verschlüsselung hingegen ist eine gänzlich neue Form
der Kryptographie, Whitfield Diffie, Martin Hellman und Ralph Merkle haben die Idee im Jahr
1976 erstmalig öffentlich eingeführt \parencite[3]{BOOK:crypto}.
Eine Übersicht über das Gebiet der Kryptographie ist in \autoref{fig:kryptologie} zu sehen.
Es ist zu bemerken, dass an oberster Stelle nicht die Kryptographie, sondern
der Oberbegriff Kryptologie zu finden ist, welche sich in die zwei großen Bereiche unterteilt:

\begin{figure}
  \centering
  \begin{tikzpicture}
    [every node/.style={draw, inner xsep=4mm, inner ysep=2mm, rounded corners=2mm, align=center}]
    \node(root) {Kryptologie};

    \node[below left=of root]  (A) {Kryptographie};
    \node[below right=of root] (B) {Kryptanalyse};

    \node[below left=of A] (C) {Symmetrische \\ Chiffren};
    \node[below=of A] (D) {Asymmetrische \\ Chiffren};
    \node[below right=of A] (E) {Protokolle};

    \coordinate (x) at ($(root.south) - (0, 0.5cm)$);
    \draw (root) -- (x);
    \draw [->] (x) -| (A);
    \draw [->] (x) -| (B);

    \coordinate (y) at ($(A.south) - (0, 0.5cm)$);
    \draw (A) -- (y);
    \draw [->] (y) -| (C);
    \draw [->] (y) -| (D);
    \draw [->] (y) -| (E);
  \end{tikzpicture}

  \caption{Die Kryptologie und ihre Untergebiete \parencite[3]{BOOK:crypto}}
  \label{fig:kryptologie}
\end{figure}

\paragraph{Kryptographie}
Die Wissenschaft eine Nachricht so zu verändern, dass ihr Sinn nur von dem Empfänger
verstanden werden kann, für den sie bestimmt ist.

\paragraph{Kryptanalyse}
Die Wissenschaft ein kryptographisches System zu analysieren mit dem Ziel mögliche
Schwachstellen aufzudecken. Die Kryptanalyse ist ein äußerst wichtiger Teil der
Kryptologie. Ohne Personen welche versuchen ein kryptographisches System zu
brechen, wird man nie herausfinden können ob das System wirklich sicher ist.
Ein starkes Kryptoverfahren sollte dem \textit{Kerckhoffs's principle} unterliegen, welches
im Jahr 1883 von Auguste Kerckhoffs postuliert wurde und von \citeauthor{BOOK:crypto}
durch folgende Definition beschrieben ist \parencite*[11]{BOOK:crypto}:

\begin{definition}[\textit{Kerckhoffs's principle}]
  \enquote{\textit{A cryptosystem should be secure even if the attacker knows all details about
      the system, with the exception of the secret key. In particular, the system should be secure when
      the attacker knows the encryption and decryption algorithms.}}
\end{definition}

\noindent
Auf den ersten Blick scheint das \textit{Kerckhoffs's principle} nicht sonderlich intuitiv.
Es sei einfach zu glauben, dass ein System
sicherer sein muss, wenn die Details der Implementierung geheim gehalten werden.
In der Regel ist dies aber nicht so. Ein Kryptoverfahren bleibt nicht für immer geheim und die
Vergangenheit hat gezeigt, dass ein System dessen geheimes Design an die Öffentlichkeit
gelangt, fast immer unsicher ist. Ein hierfür gutes Beispiel ist das Content Scrambling System (CSS)
für das Verschlüsseln von DVD-Videoinhalten. Trotz großer Bemühungen der Industrie die
Funktionsweise von CSS geheim zu halten, gelang das Design durch Reverse Code Engineering
schnell an die Öffentlichkeit. Es zeigten sich Mängel in der Implementierung,
welche das Brechen der Verschlüsselung mit sehr geringen Aufwand ermöglichten \parencite{SITE:CSS}.

\section{Symmetrische Verschlüsselung}
Denkt man an die Teilbereiche der Kryptographie, ist die Symmetrische Verschlüsselung
das wohl klassischste Beispiel. Zwei Parteien kommunizieren mit einem
Algorithmus zum Ver- und Entschlüsseln von Nachrichten und haben sich auf einen
gemeinsamen geheimen Schlüssel geeinigt. Wie es in der Literatur sehr beliebt ist,
wird die Idee der symmetrischen Verschlüsselung
mit einem einfachen Beispiel eingeführt \parencite[4-6]{BOOK:crypto}:
Zwei Parteien Alice und Bob möchten über einen unsicheren Kanal Nachrichten untereinander austauschen.
Ein unsicherer Kanal ist hierbei lediglich die Kommunikationsstrecke,
z.B. das Internet, die Luftschnittstelle bei WLAN und Mobilfunk
oder jedes andere Medium, über das sich digitale Daten übertragen lassen.

\begin{figure}[h]
  \centering

  \begin{tikzpicture}
    [box/.style={draw, align=center},
      rounded/.style={rounded corners=2mm}]

    \node[box, inner xsep=-0.1cm, inner ysep=0.3cm, ellipse]
    (K) {\small unsicherer Kanal \\ \small (z.B. Internet)};
    \node[left=3cm of K, box, rounded] (A) {\small Alice};
    \node[right=3cm of K, box, rounded] (B) {\small Bob};
    \node[above=of K, box, rounded] (O) {\small Oscar \\ \small (Angreifer)};

    \draw[->] (A) -- node[near start, above] {\small $x$} (K);
    \draw[->] (K) -- node[right] {\small $x$} (O);
    \draw[->] (K) -- node[near end, above] {\small $x$} (B);
  \end{tikzpicture}

  \caption{Kommunikation über einen unsicheren Kanal}
\end{figure}

\noindent
Es ist klar warum Alice und Bob gerne geheime Nachrichten austauschen würden. Alice möchte sich an ihrem
Bankkonto anmelden und sendet ihr Passwort zu Bob. Ein potenzieller Angreifer Oscar
soll die Passwörter von Alice nicht in Klartext mitlesen können.
In einer solchen Situation bietet die Symmetrische Verschlüsselung eine gute Lösung:
Bevor Alice ihr Passwort sendet, verschlüsselt sie es mit einem symmetrischen Algorithmus.
Bob invertiert die Verschlüsselung und erhält die unverschlüsselte Nachricht. Wurde für
die Verschlüsselung ein sicherer Algorithmus gewählt, erscheint die Nachricht für Oscar nur wie
eine zufällige Folge von Bits.
\newpage

\begin{figure}[h]
  \centering

  \begin{tikzpicture}
    [box/.style={draw, align=center},
      rounded/.style={rounded corners=2mm}]

    \node[box, inner xsep=-0.1cm, inner ysep=0.3cm, ellipse]
    (K) {\small unsicherer Kanal \\ \small (z.B. Internet)};

    \node[left=of K, box] (E) {\small Verschlüsselung \\ \small $e(x)$};
    \node[left=of E, box, rounded] (A) {\small Alice};

    \node[right=of K, box] (D) {\small Entschlüsselung \\ \small $d(y)$};
    \node[right=of D, box, rounded] (B) {\small Bob};

    \node[above=of K, box, rounded] (O) {\small Oscar \\ \small (Angreifer)};

    \node[below=of K, cylinder, draw, shape border rotate=180] (SK) {\small sicherer Kanal};

    \draw[->] (A) -- node[above] {\small $x$} (E);
    \draw[->] (E) -- node[above] {\small $y$} (K);

    \draw[<-] (B) -- node[above] {\small $x$} (D);
    \draw[<-] (D) -- node[above] {\small $y$} (K);

    \draw[<-] (O) -- node[right] {\small $y$} (K);

    \draw[->] ($(SK.top) + (2.5mm, 0)$) -| node[near end, right] {\small $k$} (E.south);
    \draw[->] (SK) -| node[near end, right] {\small $k$} (D.south);
  \end{tikzpicture}

  \caption{Kommunikation mit symmetrischer Verschlüsselung}
  \label{fig:sym-encryption}
\end{figure}

\noindent
Die Variablen $x, y$ und $k$ aus \autoref{fig:sym-encryption} haben in der
Kryptographie eine besondere Bedeutung:

\begin{itemize}[itemsep = 0pt]
  \item $x$ ist der Klartext (engl. \textit{plaintext}).
  \item $y$ ist das Chiffrat oder der Geheimtext (engl. \textit{ciphertext}).
  \item $k$ ist der Schlüssel (engl. \textit{key}).
  \item $e(\cdot)$ ist die Verschlüsselung (engl. \textit{encryption}).
  \item $d(\cdot)$ ist die Entschlüsselung (engl. \textit{decryption}),
        d.h. die Umkehrfunktion von $e$.
\end{itemize}

\noindent
Für die Symmetrische Verschlüsselung wird der geheime Schlüssel $k$ benötigt. Dieser
muss vor der Kommunikation auf einem sicheren Weg zwischen Alice und Bob verteilt werden.

\section{Modulare Arithmetik}
Fast alle kryptographischen Algorithmen, sowohl Symmetrische als auch Asymmetrische Chiffren,
basieren auf Arithmetik in einer endlichen Menge von ganzen Zahlen \parencite[13]{BOOK:crypto}.
Dies steht im Gegensatz zu der Mathematik (und dem Alltagsleben)
in der wir es gewöhnt sind in unendlichen
Mengen zu rechnen, z.B. die natürlichen Zahlen oder die reellen Zahlen. Die modulare Arithmetik,
d.h. die Division mit Rest, bietet eine gute Möglichkeit um in diesen begrenzten Mengen rechnen
zu können.

\begin{lemma}[{\cite[179-180]{BOOK:numberTheory}}]
  Folgende Aussagen über drei ganze Zahlen $a$, $b$, $m$, wobei $m > 0$, sind äquivalent:
  \begin{enumerate}[label=\roman*), itemsep = 0pt]
    \item $a$ und $b$ lassen bei Division mit Rest durch $m$ denselben Rest.
    \item Die Differenz $a - b$ ist durch $m$ teilbar.
  \end{enumerate}
\end{lemma}
\begin{proof}
  $\exists \, q_1,q_2,r_1,r_2 \in \mathbb{Z}$ \\
  Es seien $a = q_1m + r_1$ und $b = q_2m + r_2$ mit $0 \leq r_1,r_2 < m$,
  die Gleichungen, die bei Division mit Rest entstehen. \\
  \textit{i)} $\Rightarrow$ \textit{ii)}: \\
  Es gilt: $r_1 = r_2$. \\
  Zu zeigen: $m|(a-b)$.
  \begin{align*}
    a - b      & = q_1m + r_1 - q_2m + r_2  \\
    a - b      & = (q_1 - q_2)m + r_1 - r_2 \\
    \iff a - b & = (q_1 - q_2)m
  \end{align*}
  Aus der letzten Gleichung folgt: $m | (a - b)$. \\
  \textit{ii)} $\Rightarrow$ \textit{i)}: \\
  Es gilt: $m | (a - b)$. \\
  Zu zeigen: $r_1 = r_2$.
  \begin{align*}
    m                 & | (a - b)                     \\
    \Leftrightarrow m & | (q_1m + r_1) - (q_2m + r_2) \\
    \Leftrightarrow m & | (q_1 - q_2)m + (r_1 - r_2)
  \end{align*}
  $m|(q_1 - q_2)m + (r_1 - r_2)$ und $m|(q_1 - q_2)m$, weshalb gelten muss: $m|(r_1 - r_2)$.
  \footnote{Aus $a|b$ und $a|c$ folgt $a|(xb + yc)$,
    $\forall \, x,y \in \mathbb{Z}$ \parencite[23]{BOOK:numberTheory}. \\
    Also mit konkreten Werten $m|(q_1 - q_2)m + (r_1 - r_2) - (q_1 - q_2)m \Leftrightarrow m|(r_1 - r_2)$.} \\
  Entweder $r_1 - r_2$ ist ein Vielfaches von $m$ oder 0.
  Da sich $r_1$ und $r_2$ im Bereich
  $[0,m)$ befinden ist die einzige Lösung $r_1 - r_2 = 0$. Es folgt: $r_1 = r_2$.
\end{proof}

\newpage
\noindent
Nach Gauß nennt nennt man zwei Zahlen $a, b \in \mathbb{Z}$, die bei der Division durch $m$
denselben Rest ergeben, \textit{kongruent modulo m}. Anstelle der schwerfälligen
Teilbarkeitsschreibweise $m|(a - b)$
führte Gauß folgende Schreibweise ein \parencite[180]{BOOK:numberTheory}:
\begin{equation}
  \label{eq:kongruenz}
  a \equiv b \mod{m} \text{\quad oder kürzer:\quad} a \equiv b \pod{m}
\end{equation}

\begin{example}
  Es sind $a = 29$ und $m = 8$. Man schreibt
  \begin{equation*}
    29 \equiv 5 \mod{8} \text{\quad oder als Gleichung:\quad} 29 = 3 \cdot 8 + 5
  \end{equation*}
  und deshalb $8|(29 - 5)$.
\end{example}

\noindent
Die Gleichung der Modulo Rechnung hat unendlich viele Lösungen. Zu einem gegeben $m$ gibt es
beliebig viele Zahlen $a$, welche den selben Rest ergeben.
\begin{example}
  Die folgenden Zahlen erfüllen die Gleichung $a \mod{8} = 5$.
  Sie werden kongruent modulo 8 genannt:
  \begin{equation*}
    29 \equiv 21 \equiv 13 \equiv 5 \equiv -3 \equiv -11 \mod{8}
  \end{equation*}
  Elemente welche bei der Modulo Rechnung den gleichen Rest ergeben,
  können in eine Klasse zusammengefasst werden, diese nennt man dann eine Restklasse bezüglich $m$.
  Die acht Restklassen bezüglich 8 sind:
  \begin{gather*}
    \{\dots,-24,-16,-8,0,8,16,24,\dots\} \\
    \{\dots,-25,-17,-9,1,9,17,25,\dots\} \\
    \vdots                               \\
    \{\dots,-17,-9,-1,7,15,23,31,\dots\}
  \end{gather*}
\end{example}

\noindent
Alle Elemente einer Restklasse verhalten sich gleich. Zu einem gegebenen Modulo $m$ spielt es
keine Rolle, welches Element der Restklasse für eine Berechnung ausgewählt wird.
Diese Eigenschaft ist von großem Nutzen, vor allem bei Berechnungen mit großen Zahlen,
wie es in der Kryptographie oft der Fall ist.

\begin{example}[{\cite[15-16]{BOOK:crypto}}]
  Die Hauptoperation in vielen Asymmetrischen Chiffren ist die Exponentiation der Form
  $x^e \mod{m}$, wobei $x,e,m$ sehr große ganze Zahlen sind. Anhand eines Beispiels können
  zwei Formen der modularen Exponentiation gezeigt werden. Es soll das Ergebnis der
  Berechnung $3^8 \mod{7}$ ermittelt werden. Im ersten Beispiel wird das Ergebnis einfach
  ausgerechnet, und im zweiten Beispiel wird zwischen den Restklassen gewechselt:
  \begin{enumerate}[itemsep = 0pt]
    \item $3^8 = 6561 \equiv 2 \mod{7}$, weil $6561 = 937 \cdot 7 + 2$ \\
          Wir erhalten das relative große Zwischenergebnis 6561 obwohl wir wissen,
          dass das Ergebnis im Bereich $[0, 6]$ liegen muss.

    \item Es ist $3^3 = 27 \equiv -1 \mod{7}$. Man schreibt:
          \begin{equation*}
            3^8 = (3^3)^2 \cdot 3^2 \equiv (-1)^2 \cdot 9 \equiv 2 \mod{7}
          \end{equation*}
          Indem man das Zwischenergebnis $3^3 = 27$ mit einem kleineren Element aus der selben
          Restklasse ersetzt, kann das Ergebnis effizient ermittelt werden. Zwischenergebnisse
          werden nie größer als 27 und man könnte die Berechnung mit wenig Aufwand auch ohne
          Taschenrechner durchführen.
  \end{enumerate}
\end{example}

\begin{remark}
  \citeauthor{BOOK:crypto} \parencite*[16]{BOOK:crypto} haben sich in ihrem Buch darauf geeinigt,
  in der Kongruenzrelation \eqref{eq:kongruenz} ein $b$ für gewöhnlich
  so zu wählen, dass:
  \begin{equation*}
    0 \leq b < m
  \end{equation*}
  Man schreibt somit $27 \equiv 6 \pod{7}$ und nicht $27 \equiv -1 \pod{7}$
  oder $27 \equiv 13 \pod{7}$. \footnote{Mit dieser Vereinbarung ist das $b$ der Relation
    $a \equiv b \pod{m}$ eine Lösung für die Gleichung \\ $a \mod{m} = r = b$.}
  Mathematisch macht es jedoch keinen Unterschied.
\end{remark}

\newpage
\noindent
Um das bekannte Verschlüsselungsverfahren, die Cäsar-Chiffre, im nächsten Abschnitt
besser beschreiben zu können, wird an dieser Stelle der Ring $\mathbb{Z}_m$ definiert:
\begin{definition}[Der Ring $\mathbb{Z}_m$ der Reste modulo $m$]
  % https://ftp.agdsn.de/pub/mirrors/latex/dante/macros/latex/required/amscls/doc/amsthdoc.pdf#page=4&zoom=100,148,505
  \leavevmode
  \begin{enumerate}[itemsep = 0pt]
    \item Der Ring $\mathbb{Z}_m$ mit $m \in \mathbb{N}$ und $m > 1$ ist
          definiert als die Menge $\mathbb{Z}_m = \{0,1,\dots,m - 1\}$
    \item Zu je zwei Elementen $a,b \in \mathbb{Z}_m$ existiert eindeutig in $\mathbb{Z}_m$
          \begin{enumerate}[itemsep = 0pt, topsep=0pt]
            \item Eine Summe $a + b \mod{m}$
            \item Ein Produkt $a \cdot b \mod{m}$
          \end{enumerate}
  \end{enumerate}
\end{definition}

\section{Die Verschiebe- oder Cäsar-Chiffre}
Die Cäsar-Chiffre ist das vielleicht bekannteste historische Verschlüsselungsverfahren,
es wird von \citeauthor{BOOK:crypto} auf Seiten 18-19 eingeführt \parencite*{BOOK:crypto}.
Bei der Cäsar-Chiffre handelt es sich um eine spezielle Form der Buchstabensubstitution.
Jedes Zeichen im Klartext wird zur Verschlüsselung um einen bestimmten Wert
im Alphabet verschoben.
Um die Cäsar-Chiffre mathematisch zu
beschreiben, muss das zu Grunde liegende Alphabet enkodiert werden.
Eine Möglichkeit ist die fortlaufende Nummerierung der Buchstaben. Das so kodierte
lateinische Alphabet ist in \autoref{tab:encode-alph} zu sehen.

\begin{table}[h]
  \centering
  \caption{Enkodierung des lateinischen Alphabets}
  \begin{tabular}{|c|c|c|c|c|c|c|c|c|c|c|c|c|}
    \hline
    A  & B  & C  & D  & E  & F  & G  & H  & I  & J  & K  & L  & M  \\
    0  & 1  & 2  & 3  & 4  & 5  & 6  & 7  & 8  & 9  & 10 & 11 & 12 \\
    \hline
    N  & O  & P  & Q  & R  & S  & T  & U  & V  & W  & X  & Y  & Z  \\
    13 & 14 & 15 & 16 & 17 & 18 & 19 & 20 & 21 & 22 & 23 & 24 & 25 \\
    \hline
  \end{tabular}
  \label{tab:encode-alph}
\end{table}

\noindent
Würde ein Buchstabe zu weit verschoben werden, wird von vorne,
also bei dem ersten Buchstabe weitergemacht.
Sowohl Klartext- als auch Geheimtextbuchstaben sind somit Teil des Rings $\mathbb{Z}_{26}$.
Die Ver- und Entschlüsselung kann nun folgendermaßen ausgedrückt werden:

\begin{definition}[Cäsar-Chiffre]
  Es seien $x,y,k \in \mathbb{Z}_{26}$. Dann gilt:
  \begin{description}[itemsep = 0pt]
    \item[Verschlüsselung:] $y = e_k(x) \equiv x + k \mod{26}$
    \item[Entschlüsselung:] $x = d_k(y) \equiv y - k \mod{26}$
  \end{description}
\end{definition}

\newpage
\noindent
Das Verschlüsselungsverfahren funktioniert.
\begin{proof}
  Die Entschlüsselung der Verschlüsselung ergibt den Klartext. \\
  Zu zeigen: $d_k(e_k(x)) = x$.
  \begin{equation*}
    d_k(e_k(x)) = x \equiv x + k - k \equiv x \mod{26}
  \end{equation*}
\end{proof}
\begin{example}
  Es sei $k = 9$ und der Klartext:
  \begin{equation*}
    \text{\texttt{A\kern 0.2mmT\kern 0.2mmT\kern 0.2mmA\kern 0.2mmC\kern 0.2mmK}}
    = x_1,x_2,\dots,x_6 = 0,19,19,0,2,10
  \end{equation*}
  Der Geheimtext wird wie folgt berechnet:
  \begin{align*}
    e_9(0)  & \equiv 9 \mod{26}           \\
    e_9(2)  & \equiv 11 \mod{26}          \\
    e_9(10) & \equiv 19 \mod{26}          \\
    e_9(19) & \equiv 28 \equiv 2 \mod{26}
  \end{align*}
  \begin{equation*}
    y_1,y_2,\dots,y_6 = 9,2,2,9,11,19 =
    \text{\texttt{J\kern 0.2mmC\kern 0.2mmC\kern 0.2mmJ\kern 0.2mmL\kern 0.2mmT}}
  \end{equation*}
\end{example}

\noindent
Wie zu erwarten ist die Cäsar-Chiffre natürlich nicht besonders sicher.
Es gibt nur 26 verschiedene Schlüssel (wobei $k = 0$ den Klartext nicht verändert),
welche schnell alle ausprobiert werden können. Zusätzlich
haben Klartext und Geheimtext die selben statistischen Eigenschaften.
Klartextbuchstaben werden immer auf die selben Geheimtextbuchstaben abgebildet. Dies
erlaubt es eine Häufigkeitsanalyse der Buchstaben durchzuführen.