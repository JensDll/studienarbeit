\chapter{Einleitung}
Redet man heutzutage über das Thema Kryptographie, sind im Gespräch wahrscheinlich
Themen wie E-Mail-Verschlüsselung, Internetprotokolle oder Anwendungen im
Bankenwesen. Auch bekannt sind die Angriffe auf kryptographische Systeme,
wie zum Beispiel die Entzifferung der durch die Enigma-Chiffriermaschine
verschlüsselten deutschen Funksprüche während des Zweiten Weltkrieges. Es scheint als
wäre Kryptographie stark mit den modernen elektronischen Kommunikationstechniken
verbunden. Dies ist allerdings nicht so: Frühe Formen der Kryptographie
gehen zurück bis etwa 2000 v. Chr., als bereits im antiken Ägypten neben den
Standard-Hieroglyphen zusätzlich auch \enquote{geheime} Zeichen
verwendet wurden \parencite[2]{BOOK:KRYPTOGRAPHIE}. Es werden prinzipiell zwei
unterschiedliche kryptographische Verfahren unterschieden, diese sind
Symmetrische- und Asymmetrische Algorithmen. Die symmetrische Verschlüsselung ist
seit langer Zeit ein fester Bestandteil der Kryptographie, mit bekannten historischen
Verfahren wie die Cäsar-Chiffre welche bereits im antiken Rom für das Verschlüsseln von
Nachrichten verwendet wurde. Asymmetrische Verschlüsselung hingegen ist eine gänzlich neue Form
der Kryptographie, Whitfield Diffie, Martin Hellman und Ralph Merkle haben die Idee im Jahr
1976 erstmalig öffentlich eingeführt \parencite[3]{BOOK:KRYPTOGRAPHIE}.
Eine Übersicht über das Gebiet der Kryptographie ist in \autoref{fig:kryptologie}
zu sehen \parencite[3]{BOOK:KRYPTOGRAPHIE}.
Es ist zu bemerken, dass an oberster Stelle nicht die Kryptographie, sondern
der Oberbegriff Kryptologie zu finden ist, welche sich in die zwei großen Bereiche unterteilt:

\begin{figure}
\begin{center}

\begin{tikzpicture}
[every node/.style={draw, inner xsep=4mm, inner ysep=2mm, rounded corners=2mm, align=center}]
\node(root) {Kryptologie};
            
\node[below left=of root]  (A) {Kryptographie};
\node[below right=of root] (B) {Kryptanalyse};

\node[below left=of A] (C) {Symmetrische \\ Chiffren};
\node[below=of A] (D) {Asymmetrische \\ Chiffren};
\node[below right=of A] (E) {Protokolle};

\coordinate (x) at ($(root.south) - (0, 0.5cm)$);
\draw (root) -- (x);
\draw [->] (x) -| (A);
\draw [->] (x) -| (B);

\coordinate (y) at ($(A.south) - (0, 0.5cm)$);
\draw (A) -- (y);
\draw [->] (y) -| (C);
\draw [->] (y) -| (D);
\draw [->] (y) -| (E);
\end{tikzpicture}

\end{center}
\caption{Die Kryptologie und ihre Untergebiete}
\label{fig:kryptologie}
\end{figure}

\paragraph{Kryptographie}
Die Wissenschaft eine Nachricht so zu verändern, dass ihr Sinn nur von dem Empfänger
verstanden werden kann, für den sie bestimmt ist.

\paragraph{Kryptanalyse}
Die Wissenschaft ein kryptographisches System zu analysieren mit dem Ziel mögliche
Schwachstellen aufzudecken. Die Kryptanalyse ist ein äußerst wichtiger Teil der
Kryptologie. Ohne Personen welche versuchen ein kryptographisches System zu
brechen, wird man nie herausfinden können ob das System wirklich sicher ist.
Ein starkes Kryptoverfahren sollte dem \textit{Kerckhoffs's principle} unterliegen, welches
im Jahr 1883 von Auguste Kerckhoffs postuliert wurde und von \citeauthor{BOOK:KRYPTOGRAPHIE}
durch folgende Definition beschrieben ist \parencite*[11]{BOOK:KRYPTOGRAPHIE}:
\begin{definition}[\textit{Kerckhoffs's principle}]
\enquote{A cryptosystem should be secure even if the attacker knows all details about
the system, with the exception of the secret key. In particular, the system should be secure when
the attacker knows the encryption and decryption algorithms.}
\end{definition}
Auf den ersten Blick scheint das \textit{Kerckhoffs's principle} nicht sonderlich intuitiv.
Es sei einfach zu glauben, dass ein System
sicherer sein muss, wenn die Details der Implementierung geheim gehalten werden.
In der Regel ist dies aber nicht so. Ein Kryptoverfahren bleibt nicht für immer geheim und die
Vergangenheit hat gezeigt, dass ein System dessen geheimes Design an die Öffentlichkeit
gelangt, fast immer unsicher ist. Ein hierfür gutes Beispiel ist das Content Scrambling System (CSS)
für das Verschlüsseln von DVD-Videoinhalten. Trotz großer Bemühungen der Industrie die 
Funktionsweise von CSS geheim zu halten, gelang das Design durch Reverse Code Engineering
dennoch schnell an die Öffentlichkeit. Es zeigten sich Mängel in der Implementierung,
welche das Brechen der Verschlüsselung mit sehr geringen Aufwand ermöglichten \parencite{SITE:CSS}.

\section{Symmetrische Verschlüsselung}
Denkt man an die Teilbereiche der Kryptographie, ist die Symmetrische Verschlüsselung
das wohl klassischste Beispiel. Zwei Parteien kommunizieren mit einem
Algorithmus zum Ver- und Entschlüsseln von Nachrichten und haben sich auf einen
gemeinsamen geheimen Schlüssel geeinigt. Wie es in der Literatur sehr beliebt ist,
wird die Idee der symmetrischen Verschlüsselung
mit einem einfachen Beispiel eingeführt:
Zwei Parteien Alice und Bob möchten über einen unsicheren Kanal Nachrichten untereinander austauschen.
Ein unsicherer Kanal ist hierbei lediglich eine öffentliche Kommunikationsstrecke,
z.B. das Internet, eine Luftstrecke im Fall von WLAN oder Mobilfunk oder jedes andere Medium,
über das sich digitale Daten übertragen lassen. Es ist klar warum das Verschlüsseln von
Daten in einer solchen Situation erwünscht sein könnte.
