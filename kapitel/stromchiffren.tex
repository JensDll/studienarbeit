\chapter{Stromchiffren}

\begin{figure}[h]
  \begin{center}

    \begin{tikzpicture}
      [every node/.style={draw, inner xsep=4mm, inner ysep=2mm, rounded corners=2mm, align=center}]
      \node(root) {Kryptographie};

      \node[below left=of root] (A) {Symmetrische \\ Chiffren};
      \node[below=of root] (B) {Asymmetrische \\ Chiffren};
      \node[below right=of root] (C) {Protokolle};

      \node[below left=1cm and -1cm of A] (D) {Stromchiffren};
      \node[below right=1cm and -1cm of A] (E) {Blockchiffren};

      \coordinate (x) at ($(root.south) - (0, 0.5cm)$);
      \draw (root) -- (x);
      \draw [->] (x) -| (A);
      \draw [->] (x) -| (B);
      \draw [->] (x) -| (C);

      \coordinate (y) at ($(A.south) - (0, 0.5cm)$);
      \draw (A) -- (y);
      \draw [->] (y) -| (D);
      \draw [->] (y) -| (E);
    \end{tikzpicture}

  \end{center}
  \caption{Unterteilung der Symmetrischen Chiffren \parencite[29]{BOOK:crypto}}
  \label{fig:sym-ciphers-overview}
\end{figure}

\noindent
Werfen wir in \autoref{fig:sym-ciphers-overview} einen genaueren Blick auf die Algorithmen
der Kryptographie, stellen wir fest: Das Gebiet
der symmetrischen Verschlüsselungsverfahren kann aufgeteilt werden in Strom- und Blockchiffren.
In diesem Kapitel soll die Funktionsweise von Stromchiffren untersucht und der Unterschied zu
den Blockchiffren erläutert werden. Außerdem wird gezeigt welche
Rolle hierbei die Zufallszahlengeneratoren spielen.

\section{Stromchiffren und Blockchiffren}