\chapter{Stromchiffren}

\begin{figure}[h]
  \centering
  \begin{tikzpicture}
    [every node/.style={draw, inner xsep=4mm, inner ysep=2mm, rounded corners=2mm, align=center}]
    \node(root) {Kryptographie};

    \node[below left=of root] (A) {Symmetrische \\ Chiffren};
    \node[below=of root] (B) {Asymmetrische \\ Chiffren};
    \node[below right=of root] (C) {Protokolle};

    \node[below left=1cm and -1cm of A] (D) {Stromchiffren};
    \node[below right=1cm and -1cm of A] (E) {Blockchiffren};

    \coordinate[below=0.5cm of root] (x);
    \draw (root) -- (x);
    \draw [->] (x) -| (A);
    \draw [->] (x) -| (B);
    \draw [->] (x) -| (C);

    \coordinate[below=0.5cm of A] (y);
    \draw (A) -- (y);
    \draw [->] (y) -| (D);
    \draw [->] (y) -| (E);
  \end{tikzpicture}
  \caption{Unterteilung der Symmetrischen Chiffren \parencite[29]{BOOK:crypto}}
  \label{fig:sym-ciphers-overview}
\end{figure}

\noindent
Werfen wir in \autoref{fig:sym-ciphers-overview} einen genaueren Blick auf die Algorithmen
der Kryptographie, stellen wir fest: Das Gebiet
der symmetrischen Verschlüsselungsverfahren kann unterteilt werden in Strom- und Blockchiffren.
In diesem Kapitel soll die Funktionsweise von Stromchiffren untersucht und der Unterschied zu
den Blockchiffren erläutert werden. Außerdem wird gezeigt welche
Rolle hierbei die Zufallszahlengeneratoren spielen.

\section{Stromchiffren und Blockchiffren}
Die symmetrischen Verschlüsselungsverfahren sind unterteilt in Strom- und Blockchiffren.
Während beide Verfahren das Ziel haben Informationen zu verschlüsseln, ist die jeweilige
Methode eine unterschiedliche. Stromchiffren verschlüsseln jedes Bit im Klartext einzeln,
während Blockchiffren pro Durchlauf mehrere Bits verschlüsseln können.
\autoref{fig:stream-vs-block-cipher} zeigt diesen prinzipiellen Unterschied, für den Fall, dass
$n$ Bits verschlüsselt werden sollen.

\begin{figure}[h]
  \centering
  \begin{tikzpicture}
    [box/.style={draw, inner xsep=4mm, inner ysep=2mm}]

    \node[box] (stream) {Stromchiffre};
    \node[font=\small, above=of stream] (k1) {$k$};
    \node[below=1pt of stream] {(a)};

    \node[box, right=5cm of stream] (block) {Blockchiffre};
    \node[font=\small, above=of block] (k2) {$k$};
    \node[below=1pt of block] {(b)};

    \draw[->] (k1) to (stream);
    \draw[<-] (stream.west) to node[pos=1, above, font=\small] {$x_1,x_2,\dots,x_n$} ++(-1.5cm,0);
    \draw[->] (stream.east) to node[pos=1, above, font=\small] {$y_1,y_2,\dots,y_n$} ++(1.5cm,0);

    \draw[->] (k2) to (block);
    \draw[<-] (block.west) to
    node[pos=1, above, font=\small, align=center] {
      $x_1$ \\
      $x_2$ \\
      \vdots \\
      $x_n$
    } ++(-1.5cm,0);
    \draw[->] (block.east) to
    node[pos=1, above, font=\small, align=center] {
      $y_1$ \\
      $y_2$ \\
      \vdots \\
      $y_n$
    } ++(1.5cm,0);
  \end{tikzpicture}
  \caption{Das Prinzip der Verschlüsselung von $n$ Bits mittels Strom- (a) und
    Blockchiffre (b) \parencite[30]{BOOK:crypto}}
  \label{fig:stream-vs-block-cipher}
\end{figure}

\noindent
Blockchiffren arbeiten mit Datenblöcken fester Länge, wobei ein Großteil der
relevanten Algorithmen eine Eingangsweite von 64 oder 128 Bits besitzen.
Prominente Beispiele sind Verfahren wie der Data Encryption Standard
(DES, Blocklänge 64 Bit, \cite[55-58]{BOOK:crypto}) oder der Advanced Encryption Standard
(AES, Blocklänge 128 Bit, \cite[87-90]{BOOK:crypto}).
Generell lassen sich die folgenden drei Punkte zusammenfassen \parencite[31]{BOOK:crypto}:

\begin{enumerate}[itemsep = 0pt]
  \item Blockchiffren werden in der Praxis, vor allem für die Verschlüsselung im Internet,
        häufiger eingesetzt als Stromchiffren.
  \item Stromchiffren sind oft klein und schnell und deshalb attraktiv für Anwendungen,
        bei denen vergleichsweise wenig Rechenleistung zur Verfügung steht, beispielsweise bei
        Mobilgeräten oder anderen eingebetteten Systemen. Ein bekanntes Verfahren ist der
        A5/1 Algorithmus, welcher Teil des GSM-Mobilfunkstandards ist und die Gesprächsdaten
        an der Luftschnittstelle verschlüsselt. A5/1 gilt aufgrund der kurzen Schlüssellänge von
        64 Bit nicht mehr als uneingeschränkt sicher, erschwert aber dennoch das einfache Abhören.
  \item In der Vergangenheit galt der generelle Gedanke, dass Stromchiffren
        effizienter seien als Blockchiffren.
        Im Allgemeinen gilt diese Annahme heutzutage jedoch nicht mehr. Moderne Verfahren wie AES
        können sowohl in Hardware als auch Software sehr effizient implementiert werden.
\end{enumerate}

\newpage
\section{Die Ver- und Entschlüsselung mit Stromchiffren}
Wie oben erwähnt, verschlüsseln Stromchiffren jedes Bit im Klartext einzeln. Hierfür
wird durch einen Schlüssel ein geheimer Bitstrom errechnet, welcher paarweise mit den Bits
des Klartextes kombiniert wird. Die Ver- und Entschlüsselung ist verblüffend einfach, es handelt
sich in beide Richtungen um eine einfache Addition im Ring $\mathbb{Z}_2$.

\begin{definition}[{Ver- und Entschlüsselung mit Stromchiffren, \cite[31]{BOOK:crypto}}]
  Es seien $x_i,y_i,s_i \in \{0,1\}$ die einzelnen Bits aus Klartext, Geheimtext und Schlüs\-selstrom.
  Es gilt:
  \begin{description}[itemsep = 0pt]
    \item[Verschlüsselung:] $y_i = e_{s_i}(x_i) \equiv x_i + s_i \mod{2}$
    \item[Entschlüsselung:] $x_i = d_{s_i}(y_i) \equiv y_i + s_i \mod{2}$
  \end{description}
\end{definition}

\noindent
Betrachtet man die Ver- und Entschlüsselungsfunktion, fallen drei Aspekte auf welche
besprochen werden müssen \parencite[31-34]{BOOK:crypto}:

\begin{enumerate}[itemsep = 0pt]
  \item Warum ist Verschlüsselung und Entschlüsselung die selbe Funktion?
  \item Warum ist einfache Addition im Ring $\mathbb{Z}_2$ eine gute Verschlüsselung?
  \item Was sind die Eigenschaften der Schlüsselstrombits $s_i$?
\end{enumerate}

\paragraph{Warum ist Verschlüsselung und Entschlüsselung die selbe Funktion?}\mbox{}\\
Bis auf spezielle Kürzungsregeln erlauben Kongruenzen (ganz analog wie mit Gleichungen)
das Ausführen der elementaren Rechenoperationen \parencite[181-183]{BOOK:numberTheory}.
Mit diesem Wissen kann die Verschlüsselungsfunktion durch einfaches Umformen in die
Entschlüsselungsfunktion überführt werden.
\begin{proof}
  \begin{align*}
    y_i  & \equiv x_i + s_i \mod{2}  \\
    -x_i & \equiv -y_i + s_i \mod{2} \\
    x_i  & \equiv y_i + s_i \mod{2}
  \end{align*}
\end{proof}
\noindent
Im letzten Schritt wird erneut von dem Wechsel innerhalb der Restklasse Gebrauch gemacht,
denn es gilt $-1 \equiv 1 \mod{2}$.

\paragraph{Warum ist einfache Addition im Ring $\mathbb{Z}_2$ eine gute Verschlüsselung?}\mbox{}\\
Das Rechnen in $\mathbb{Z}_2$ liefert aufgrund der Division mit Rest nur Ergebnisse in
der Menge $\{0,1\}$. Dieses Verhalten ist sehr hilfreich, denn es ermöglicht das Ausdrücken der
Arithmetik durch einfache boolesche Algebra.
Betrachtet man die Wahrheitstabelle \ref{tab:truth-table-addition-mod-2}
der Addition in $\mathbb{Z}_2$, kann sofort eine weitere Beobachtung gemacht werden:
Die Addition modulo 2 ist äquivalent zu der Exklusiv-Oder-Verknüpfung durch ein XOR-Gatter.

\begin{table}[h]
  \centering
  \caption{Wahrheitstabelle der Addition modulo 2}
  \begin{tabular}{cc|c}
    $x_i$ & $s_i$ & $y_i \equiv x_i + s_i \mod{2}$ \\ \hline
    0     & 0     & 0                              \\
    0     & 1     & 1                              \\
    1     & 0     & 1                              \\
    1     & 1     & 0                              \\
  \end{tabular}
  \label{tab:truth-table-addition-mod-2}
\end{table}

\noindent
Das XOR-Gatter spielt eine wesentliche Rolle in vielen kryptographischen Verfahren.
Was es so besonders macht und es von anderen Logikgattern
wie beispielsweise der Und-Verknüpfung unterscheidet, soll jetzt untersucht werden.
Angenommen es soll das Klartextbit $x_i = 0$ verschlüsselt werden. In der Wahrheitstabelle
des XOR-Gatters befinden wir uns demnach in der ersten oder zweiten Zeile:

\begin{table}[h]
  \centering
  \caption{Wahrheitstabelle der Exklusiv-Oder-Verknüpfung}
  \begin{tabular}{cc|c}
    $x_i$      & $s_i$      & $y_i = x_i \oplus s_i$ \\ \hline
    \textbf{0} & \textbf{0} & \textbf{0}             \\
    \textbf{0} & \textbf{1} & \textbf{1}             \\
    1          & 0          & 1                      \\
    1          & 1          & 0                      \\
  \end{tabular}
  \label{tab:truth-table-xor}
\end{table}

\noindent
Je nach Schlüsselbit $s_i$ ist das Geheimtextbit $y_i$ gegeben als
$y_i = 0 \oplus 0 \vee 0 \oplus 1 = 0 \vee 1$.
Verhalten sich die Schlüsselbit unvorhersehbar , d.h. sie sind mit genau 50 prozentiger
Wahrscheinlichkeit entweder null oder eins, ist es nur durch das Chiffrat nicht möglich auf
den Klartext zu schließen. Gleichermaßen kann argumentiert werden, sei $x_i = 1$.
Zu jedem Zeitpunkt hat ein Angreifer nur eine 50 prozentige Chance den richtigen Klartext zu erraten.
Diese Symmetrie unterscheidet das XOR-Gatter von anderen Logikgattern.
\autoref{fig:enc-dec-stream-cipher} zeigt den Ver- und Entschlüsselungsprozess
bei Stromchiffren.
\newpage

\begin{figure}
  \centering
  \begin{tikzpicture}
    \node[XOR, scale=2] (xor-e) {};
    \node[above=of xor-e] (s1) {$s_i$};

    \node[unsecure channel, right=1.5cm of xor-e]
    (K) {\small unsicherer Kanal \\ \small (z.B. Internet)};

    \node[XOR, scale=2, right=1.5cm of K] (xor-d) {};
    \node[above=of xor-d] (s2) {$s_i$};

    \draw[<-] (xor-e.west) to node[pos=1, above] {$x_i$} ++(-1.5cm,0);
    \draw[->] (s1) to (xor-e);
    \draw[->] (xor-e) to node[above] {$y_i$} (K);

    \draw[->] (K) to node[above] {$y_i$} (xor-d);
    \draw[->] (s2) to (xor-d);
    \draw[->] (xor-d) to node[pos=1, above] {$x_i$} ++(1.5cm,0);
  \end{tikzpicture}
  \caption{Der Ver- und Entschlüsselungsprozess bei Stromchiffren}
  \label{fig:enc-dec-stream-cipher}
\end{figure}

\noindent
In einem Beispiel soll jetzt ein einfacher Informationsaustausch demonstriert werden:

\newcommand{\streamencryption}{
  \begin{aligned}
    x_0,\dots,x_7 = 01010 & 000_2 = 80_{10} = \text{\texttt{P}}  \\
    \oplus                &                                      \\
    s_0,\dots,s_7 = 00111 & 010_2                                \\
    y_0,\dots,y_7 = 01101 & 010_2 = 106_{10} = \text{\texttt{j}}
  \end{aligned}
}

\newcommand{\streamdecryption}{
  \begin{aligned}
    y_0,\dots,y_7 = 01101 & 010_2 = 106_{10} = \text{\texttt{j}} \\
    \oplus                &                                      \\
    s_0,\dots,s_7 = 00111 & 010_2                                \\
    x_0,\dots,x_7 = 01010 & 000_2 = 80_{10} = \text{\texttt{P}}
  \end{aligned}
}

\newcommand{\streamarrow}{\tikz{\draw[->] (0,0) to
    node[above] {$\text{\texttt{j}} = 01101010_2$} (2,0);}}

\begin{example}
  Alice möchte Bob eine Nachricht senden. Sie verschickt den Buchstaben \texttt{P},
  wobei dieser zur Verschlüsselung in
  ASCII-Zeichenkodierung angegeben ist $\text{\texttt{P}} = 80_{10} = 01010000_2$. Der
  Schlüsselstrom wird außerdem angegeben, es seien $(s_0,s_1,\dots,s_7) = 00111010_2$.

  \begin{table*}[h]
    \centering
    \begin{tabular}{lc}
      $\streamencryption$              & \textbf{Alice} \\
                                       &                \\
      \multicolumn{1}{c}{\streamarrow} & \textbf{Oscar} \\
                                       &                \\
      $\streamdecryption$              & \textbf{Bob}
    \end{tabular}
  \end{table*}

  \noindent
  Vor der Übertragung wird der Großbuchstabe \texttt{P} durch die XOR-Verknüpfung
  in den Kleinbuchstaben \texttt{j}
  umgewandelt. Der Gegenspieler Oscar kann mit dieser Informationen aufgrund der oben beschriebenen
  Eigenschaften nur wenig anfangen.
\end{example}

\noindent
Stromchiffren erscheinen fast zu gut, um wahr zu sein, doch wie gewohnt sind die Dinge in
der Realität nicht immer so einfach. Es bleibt die Letzte der drei Fragestellungen zu
beantworten.
\newpage

\paragraph{Was sind die Eigenschaften der Schlüsselstrombits?}\mbox{}\\
Die Sicherheit von Stromchiffren hängt vollständig von der Qualität des Schlüsselstroms ab.
Es folgt hieraus, dass die Generierung dieser Bitfolge
($s_1,s_2,\dots,s_n$) die zentrale Fragestellung der Verschlüsselungsmethode bildet.
Es wurde gezeigt, dass der Schlüsselstrom wie eine zufällige
Folge von Bits aussehen muss. Im nachfolgenden Abschnitt werden deshalb Zufallszahlen
diskutiert und wie diese generiert werden können.

\section{Zufallszahlengeneratoren}