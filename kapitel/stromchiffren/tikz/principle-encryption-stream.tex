\begin{figure}[h]
  \centering
  \begin{tikzpicture}
    [box/.style={draw, inner xsep=4mm, inner ysep=2mm}]

    \node[box] (stream) {Stromchiffre};
    \node[font=\small, above=of stream] (k1) {$k$};
    \node[below=1pt of stream] {(a)};

    \node[box, right=5cm of stream] (block) {Blockchiffre};
    \node[font=\small, above=of block] (k2) {$k$};
    \node[below=1pt of block] {(b)};

    \draw[->] (k1) to (stream);
    \draw[<-] (stream.west) to node[pos=1, above, font=\small] {$x_1,x_2,\dots,x_n$} ++(-1.5cm,0);
    \draw[->] (stream.east) to node[pos=1, above, font=\small] {$y_1,y_2,\dots,y_n$} ++(1.5cm,0);

    \draw[->] (k2) to (block);
    \draw[<-] (block.west) to
    node[pos=1, above, font=\small, align=center] {
      $x_1$ \\
      $x_2$ \\
      \vdots \\
      $x_n$
    } ++(-1.5cm,0);
    \draw[->] (block.east) to
    node[pos=1, above, font=\small, align=center] {
      $y_1$ \\
      $y_2$ \\
      \vdots \\
      $y_n$
    } ++(1.5cm,0);
  \end{tikzpicture}
  \caption{Das Prinzip der Verschlüsselung von $n$ Bits mittels Strom- (a) und
    Blockchiffre (b) \parencite[30]{BOOK:crypto}}
  \label{fig:stream-vs-block-cipher}
\end{figure}