\newenvironment{abstractpage}
{\cleardoublepage\vspace*{\fill}\thispagestyle{empty}}
{\vfill\cleardoublepage}
\newenvironment{myabstract}[1]
{\bigskip\selectlanguage{#1}
  \begin{center}
    \bfseries\abstractname
  \end{center}}
{\par\bigskip}

\begin{abstractpage}
  \begin{myabstract}{german}
    Diese Arbeit beschäftigt sich mit dem Entwurf eines steganografischen Algorithmus, um
    Nachrichten in einer Bilddatei zu verstecken.
    Die Prozedur verwendet ein \glsentrylong{lsb} Einbettungsverfahren und es
    kann experimentell gezeigt werden, dass durch pseudozufällige
    Pixelauswahl bis zu einer oberen Schranke ein
    steganografisch gutes Ergebnis erreicht wird.
    Das Verfahren wird als Teil einer Anwendung verwendet,
    um einen sicheren Nachrichtenaustausch über in Bildern versteckten
    Informationen zu ermöglichen. Am Beispiel von
    symmetrischen und asymmetrischen Chiffren werden Begriffe und Ideen
    aus dem Bereich der Kryptografie eingeführt. Wir gehen auf
    Strom- und Blockchiffren ein und beweisen warum die
    RSA-Verschlüsselung funktioniert und sicher ist.
    Die Sicherheitsdienste der Anwendung können anschließend ermittelt
    und bewertet werden.
  \end{myabstract}
  \begin{myabstract}{english}
    We show how to implement an algorithm for data hiding in colour images,
    using a least significant bit embedding technique.
    This type of secret sharing through any set of cover media
    is called steganography.
    Considering some upper bound message length,
    experiments show that high-quality stego-images can be achieved
    by selecting a pseudorandom distribution of pixel positions.
    Using the algorithm as part of an application we can describe a system
    for securely sharing secret information through messages embedded in images.
    We explain some of the main techniques in cryptography, with chapters
    addressing symmetric as well as asymmetric ciphers.
    In particular, we introduce stream and block ciphers and proof why the
    RSA scheme works and is secure. The security services of the
    application can then be defined and afterwards evaluated.
  \end{myabstract}
\end{abstractpage}