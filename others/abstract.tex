\newenvironment{abstractpage}
{\cleardoublepage\vspace*{\fill}\thispagestyle{empty}}
{\vfill\cleardoublepage}
\newenvironment{myabstract}[1]
{\bigskip\selectlanguage{#1}
  \begin{center}
    \bfseries\abstractname
  \end{center}}
{\par\bigskip}

\begin{abstractpage}
  \begin{myabstract}{german}
    In dieser Arbeit werden eine Mehrheit von Begriffen und Ideen aus dem Bereich
    der Kryptografie eingeführt. Es wird sowohl auf symmetrische als auch asymmetrische
    Chiffren eingegangen, wobei insbesondere der Entwurf von Stromchiffren
    untersucht und die Korrektheit der RSA-Verschlüsselung bewiesen wird.
    Es finden Untersuchungen zur Steganografie statt und es wird ein Algorithmus
    vorgestellt, welcher eine beliebig lange Nachricht in einer Bilddatei
    versteckt. Das Ergebnis zeigt, dass auch ein vergleichsweise einfaches Verfahren
    eine gute steganografische Qualität liefert und Nachrichten
    mit einer Länge von bis zu 50\,\% der maximalen Bildkapazität in den meisten Fällen
    nur schwer zu erkennen sind.
  \end{myabstract}
  \begin{myabstract}{english}
    Test
  \end{myabstract}
\end{abstractpage}