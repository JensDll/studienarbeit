\newenvironment{abstractpage}
{\cleardoublepage\vspace*{\fill}\thispagestyle{empty}}
{\vfill\cleardoublepage}
\newenvironment{myabstract}[1]
{\bigskip\selectlanguage{#1}
  \begin{center}
    \bfseries\abstractname
  \end{center}}
{\par\bigskip}

\begin{abstractpage}
  \begin{myabstract}{german}
    Diese Arbeit beschäftigt sich mit dem Entwurf eines steganografischen Algorithmus, um
    Nachrichten in einer Bilddatei zu verstecken.
    Der Algorithmus verwendet ein \glsentrylong{lsb} Einbettungsverfahren. Es
    kann experimentell gezeigt werden, dass durch pseudozufällige
    Pixelauswahl bis zu einer oberen Schranke ein
    steganografisch gutes Ergebnis erreicht wird.
    Das Verfahren wird als Teil einer Anwendung verwendet,
    um einen sicheren Nachrichtenaustausch über in Bildern versteckten
    Informationen zu ermöglichen. Am Beispiel von
    symmetrischen und asymmetrischen Chiffren werden Begriffe und Ideen
    aus dem Bereich der Kryptografie eingeführt. Wir gehen auf
    Strom- und Blockchiffren ein und beweisen warum die
    RSA-Verschlüsselung funktioniert und sicher ist.
    Die Sicherheitsdienste der Anwendung können anschließend ermittelt
    und bewertet werden.
  \end{myabstract}
  \begin{myabstract}{english}

  \end{myabstract}
\end{abstractpage}